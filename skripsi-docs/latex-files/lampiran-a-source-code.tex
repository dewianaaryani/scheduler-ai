\appendix
\section{Repositori Source Code}

\subsection{Struktur Proyek}

Aplikasi Scheduler AI dikembangkan menggunakan arsitektur Next.js 15 dengan struktur direktori yang terorganisir sebagai berikut:

\begin{lstlisting}[language=bash, caption=Struktur Direktori Utama]
scheduler-ai/
├── app/                    # Next.js App Router
│   ├── (logged-in)/       # Routes yang memerlukan autentikasi
│   │   ├── (app-layout)/  # Layout dengan sidebar
│   │   │   ├── dashboard/ # Halaman dashboard utama
│   │   │   ├── calendar/  # Manajemen kalender
│   │   │   ├── goals/     # Manajemen tujuan
│   │   │   ├── scheduler/ # AI scheduling assistant
│   │   │   └── settings/  # Pengaturan pengguna
│   │   └── onboarding/    # Proses onboarding pengguna baru
│   ├── api/               # API Routes
│   │   ├── auth/          # Autentikasi NextAuth.js
│   │   ├── dashboard/     # Dashboard data endpoints
│   │   ├── goals/         # Goals management API
│   │   ├── schedules/     # Schedule management API
│   │   ├── calendar/      # Calendar view API
│   │   ├── ai/            # AI integration endpoints
│   │   ├── ai-chat/       # AI chat functionality
│   │   ├── claude/        # Claude AI integration
│   │   ├── upload/        # File upload handling
│   │   └── user/          # User management API
│   ├── components/        # React components
│   ├── lib/               # Utility libraries
│   └── hooks/             # Custom React hooks
├── components/            # Shared UI components (shadcn/ui)
├── prisma/                # Database schema dan migrations
├── public/                # Static assets
└── skripsi-docs/          # Dokumentasi skripsi
\end{lstlisting}

\subsection{Teknologi Stack}

\begin{table}[H]
\centering
\caption{Stack Teknologi yang Digunakan}
\begin{tabular}{|l|l|l|}
\hline
\textbf{Kategori} & \textbf{Teknologi} & \textbf{Versi} \\
\hline
Frontend Framework & Next.js & 15.0.0 \\
React & React & 18.3.1 \\
TypeScript & TypeScript & 5.6.0 \\
Styling & Tailwind CSS & 3.4.1 \\
UI Components & Radix UI & Various \\
Database & PostgreSQL & 14+ \\
ORM & Prisma & 5.20.0 \\
Authentication & NextAuth.js & 5.0.0-beta \\
AI Integration & OpenAI API & Latest \\
Claude AI & Anthropic API & Latest \\
File Storage & Supabase Storage & Latest \\
State Management & React Hooks & Built-in \\
Form Handling & React Hook Form & 7.53.0 \\
Validation & Zod & 3.23.8 \\
\hline
\end{tabular}
\end{table}

\subsection{Core Source Code Components}

\subsubsection{Database Schema (Prisma)}

\begin{lstlisting}[language=JavaScript, caption=prisma/schema.prisma - User Model]
model User {
  id            String     @id @default(cuid())
  name          String?    @db.VarChar(100)
  email         String     @unique
  emailVerified DateTime?
  preferences   Json?
  image         String?
  accounts      Account[]
  sessions      Session[]
  goals         Goal[]
  schedules     Schedule[]

  createdAt DateTime @default(now())
  updatedAt DateTime @updatedAt
}
\end{lstlisting}

\begin{lstlisting}[language=JavaScript, caption=prisma/schema.prisma - Goal Model]
model Goal {
  id          String     @id @default(uuid())
  userId      String
  title       String     @db.VarChar(100)
  description String     @db.VarChar(500)
  startDate   DateTime   @default(now())
  endDate     DateTime
  emoji       String     @db.VarChar(20)
  status      StatusGoal
  createdAt   DateTime   @default(now())
  updatedAt   DateTime   @updatedAt
  user        User       @relation(fields: [userId], references: [id])
  schedules   Schedule[]
}

enum StatusGoal {
  ACTIVE
  COMPLETED
  ABANDONED
}
\end{lstlisting}

\subsubsection{API Routes}

\begin{lstlisting}[language=TypeScript, caption=app/api/goals/route.ts - Goals API]
import { auth } from "@/app/lib/auth";
import { db } from "@/app/lib/db";
import { NextRequest, NextResponse } from "next/server";

export async function GET() {
  try {
    const session = await auth();
    if (!session?.user?.id) {
      return NextResponse.json({ error: "Unauthorized" }, { status: 401 });
    }

    const goals = await db.goal.findMany({
      where: { userId: session.user.id },
      include: {
        schedules: {
          select: {
            id: true,
            status: true,
            percentComplete: true,
          },
        },
      },
      orderBy: { createdAt: "desc" },
    });

    const goalsWithProgress = goals.map((goal) => {
      const totalSchedules = goal.schedules.length;
      const completedSchedules = goal.schedules.filter(
        (schedule) => schedule.status === "COMPLETED"
      ).length;
      
      const percentComplete = totalSchedules > 0 
        ? Math.round((completedSchedules / totalSchedules) * 100) 
        : 0;

      return {
        ...goal,
        percentComplete,
      };
    });

    return NextResponse.json(goalsWithProgress);
  } catch (error) {
    console.error("Error fetching goals:", error);
    return NextResponse.json(
      { error: "Internal server error" },
      { status: 500 }
    );
  }
}
\end{lstlisting}

\subsubsection{React Components}

\begin{lstlisting}[language=TypeScript, caption=app/components/goal-summary.tsx]
"use client";

import { useState, useEffect } from "react";
import { Card, CardContent, CardHeader, CardTitle } from "@/components/ui/card";
import { Badge } from "@/components/ui/badge";
import { Progress } from "@/components/ui/progress";

interface Goal {
  id: string;
  title: string;
  description: string;
  emoji: string;
  status: "ACTIVE" | "COMPLETED" | "ABANDONED";
  percentComplete: number;
  startDate: string;
  endDate: string;
}

export default function GoalSummary() {
  const [goals, setGoals] = useState<Goal[]>([]);
  const [loading, setLoading] = useState(true);

  useEffect(() => {
    fetchGoals();
  }, []);

  const fetchGoals = async () => {
    try {
      const response = await fetch("/api/goals");
      if (response.ok) {
        const data = await response.json();
        setGoals(data);
      }
    } catch (error) {
      console.error("Error fetching goals:", error);
    } finally {
      setLoading(false);
    }
  };

  if (loading) {
    return <div>Loading goals...</div>;
  }

  return (
    <div className="grid gap-4">
      {goals.map((goal) => (
        <Card key={goal.id} className="cursor-pointer hover:shadow-md">
          <CardHeader className="pb-3">
            <CardTitle className="flex items-center gap-2">
              <span className="text-2xl">{goal.emoji}</span>
              <span className="flex-1">{goal.title}</span>
              <Badge 
                variant={goal.status === "ACTIVE" ? "default" : "secondary"}
              >
                {goal.status}
              </Badge>
            </CardTitle>
          </CardHeader>
          <CardContent>
            <p className="text-muted-foreground mb-4">{goal.description}</p>
            <div className="space-y-2">
              <div className="flex justify-between text-sm">
                <span>Progress</span>
                <span>{goal.percentComplete}%</span>
              </div>
              <Progress value={goal.percentComplete} className="h-2" />
            </div>
          </CardContent>
        </Card>
      ))}
    </div>
  );
}
\end{lstlisting}

\subsubsection{Custom Hooks}

\begin{lstlisting}[language=TypeScript, caption=app/hooks/useGoals.ts]
import { useState, useEffect } from "react";

interface Goal {
  id: string;
  title: string;
  description: string;
  emoji: string;
  status: string;
  percentComplete: number;
  startDate: string;
  endDate: string;
}

export function useGoals() {
  const [goals, setGoals] = useState<Goal[]>([]);
  const [loading, setLoading] = useState(true);
  const [error, setError] = useState<string | null>(null);

  const fetchGoals = async () => {
    try {
      setLoading(true);
      const response = await fetch("/api/goals");
      
      if (!response.ok) {
        throw new Error("Failed to fetch goals");
      }
      
      const data = await response.json();
      setGoals(data);
      setError(null);
    } catch (err) {
      setError(err instanceof Error ? err.message : "Unknown error");
    } finally {
      setLoading(false);
    }
  };

  const createGoal = async (goalData: any) => {
    try {
      const response = await fetch("/api/goals", {
        method: "POST",
        headers: {
          "Content-Type": "application/json",
        },
        body: JSON.stringify({ dataGoals: goalData }),
      });

      if (!response.ok) {
        throw new Error("Failed to create goal");
      }

      await fetchGoals(); // Refresh goals list
      return true;
    } catch (err) {
      setError(err instanceof Error ? err.message : "Unknown error");
      return false;
    }
  };

  useEffect(() => {
    fetchGoals();
  }, []);

  return {
    goals,
    loading,
    error,
    fetchGoals,
    createGoal,
  };
}
\end{lstlisting}

\subsubsection{AI Integration}

\begin{lstlisting}[language=TypeScript, caption=app/api/ai-chat/route.ts - AI Goal Processing]
import { auth } from "@/app/lib/auth";
import { db } from "@/app/lib/db";
import { NextRequest, NextResponse } from "next/server";
import OpenAI from "openai";

const openai = new OpenAI({
  apiKey: process.env.OPENAI_API_KEY,
});

export async function POST(request: NextRequest) {
  try {
    const session = await auth();
    if (!session?.user?.id) {
      return NextResponse.json({ error: "Unauthorized" }, { status: 401 });
    }

    const { initialValue, title, description, startDate, endDate } = 
      await request.json();

    const prompt = `
    Analisis input pengguna berikut dan ekstrak informasi goal:
    Input: "${initialValue}"
    
    Ekstrak dan format sebagai JSON dengan field:
    - title: judul goal yang jelas
    - description: deskripsi detail
    - startDate: tanggal mulai (YYYY-MM-DD)
    - endDate: tanggal target selesai (YYYY-MM-DD)
    - emoji: emoji yang sesuai
    - steps: array langkah-langkah untuk mencapai goal
    
    Berikan response dalam format JSON yang valid.
    `;

    const completion = await openai.chat.completions.create({
      model: "gpt-3.5-turbo",
      messages: [
        {
          role: "system",
          content: "Anda adalah assistant yang membantu menganalisis dan memformat goal planning.",
        },
        {
          role: "user",
          content: prompt,
        },
      ],
      temperature: 0.7,
    });

    const aiResponse = completion.choices[0]?.message?.content;
    
    if (!aiResponse) {
      throw new Error("No response from AI");
    }

    // Parse AI response
    const parsedGoal = JSON.parse(aiResponse);

    return NextResponse.json({
      success: true,
      data: parsedGoal,
    });
  } catch (error) {
    console.error("Error in AI chat:", error);
    return NextResponse.json(
      { error: "Failed to process AI request" },
      { status: 500 }
    );
  }
}
\end{lstlisting}

\subsubsection{Utility Functions}

\begin{lstlisting}[language=TypeScript, caption=app/lib/validation.ts - Data Validation]
// Validation functions to prevent database constraint violations
export const validateGoalTitle = (title: string | null | undefined): string => {
  if (!title) return "";
  return title.length > 100 ? title.slice(0, 97) + "..." : title;
};

export const validateGoalDescription = (description: string | null | undefined): string => {
  if (!description) return "";
  return description.length > 500 ? description.slice(0, 497) + "..." : description;
};

export const validateEmoji = (emoji: string | null | undefined): string => {
  if (!emoji) return "🎯";
  return emoji.length > 20 ? emoji.slice(0, 20) : emoji;
};

export const validateScheduleTitle = (title: string | null | undefined): string => {
  if (!title) return "";
  return title.length > 100 ? title.slice(0, 97) + "..." : title;
};

export const validateScheduleDescription = (description: string | null | undefined): string => {
  if (!description) return "";
  return description.length > 500 ? description.slice(0, 497) + "..." : description;
};

export const validateScheduleNotes = (notes: string | null | undefined): string => {
  if (!notes) return "";
  return notes.length > 500 ? notes.slice(0, 497) + "..." : notes;
};

// Date validation
export const validateDateRange = (startDate: Date, endDate: Date): boolean => {
  return startDate <= endDate;
};

// Time validation for schedules
export const validateTimeRange = (startTime: Date, endTime: Date): boolean => {
  return startTime < endTime;
};
\end{lstlisting}

\subsection{Configuration Files}

\subsubsection{Next.js Configuration}

\begin{lstlisting}[language=TypeScript, caption=next.config.ts]
import type { NextConfig } from "next";

const nextConfig: NextConfig = {
  experimental: {
    serverActions: {
      allowedOrigins: ["localhost:3000"],
    },
  },
  images: {
    domains: ["lh3.googleusercontent.com", "avatars.githubusercontent.com"],
  },
  env: {
    NEXTAUTH_URL: process.env.NEXTAUTH_URL,
    NEXTAUTH_SECRET: process.env.NEXTAUTH_SECRET,
  },
};

export default nextConfig;
\end{lstlisting}

\subsubsection{TypeScript Configuration}

\begin{lstlisting}[language=JSON, caption=tsconfig.json]
{
  "compilerOptions": {
    "lib": ["dom", "dom.iterable", "es6"],
    "allowJs": true,
    "skipLibCheck": true,
    "strict": true,
    "noEmit": true,
    "esModuleInterop": true,
    "module": "esnext",
    "moduleResolution": "bundler",
    "resolveJsonModule": true,
    "isolatedModules": true,
    "jsx": "preserve",
    "incremental": true,
    "plugins": [
      {
        "name": "next"
      }
    ],
    "baseUrl": ".",
    "paths": {
      "@/*": ["./*"]
    }
  },
  "include": ["next-env.d.ts", "**/*.ts", "**/*.tsx", ".next/types/**/*.ts"],
  "exclude": ["node_modules"]
}
\end{lstlisting}

\subsubsection{Package Dependencies}

\begin{lstlisting}[language=JSON, caption=package.json - Key Dependencies]
{
  "name": "scheduler-ai",
  "version": "0.1.0",
  "private": true,
  "scripts": {
    "dev": "next dev --turbopack",
    "build": "next build",
    "start": "next start",
    "lint": "next lint",
    "db:generate": "prisma generate",
    "db:migrate": "prisma migrate dev",
    "db:seed": "prisma db seed",
    "db:studio": "prisma studio"
  },
  "dependencies": {
    "@anthropic-ai/sdk": "^0.27.3",
    "@auth/prisma-adapter": "^2.7.2",
    "@hookform/resolvers": "^3.9.1",
    "@prisma/client": "^5.20.0",
    "@radix-ui/react-avatar": "^1.1.1",
    "@radix-ui/react-dialog": "^1.1.2",
    "@radix-ui/react-dropdown-menu": "^2.1.2",
    "@radix-ui/react-form": "^0.1.0",
    "@radix-ui/react-popover": "^1.1.2",
    "@radix-ui/react-progress": "^1.1.0",
    "@radix-ui/react-select": "^2.1.2",
    "@radix-ui/react-tabs": "^1.1.1",
    "@supabase/supabase-js": "^2.45.4",
    "class-variance-authority": "^0.7.0",
    "clsx": "^2.1.1",
    "lucide-react": "^0.454.0",
    "next": "15.0.0",
    "next-auth": "5.0.0-beta.21",
    "openai": "^4.67.1",
    "react": "^18.3.1",
    "react-dom": "^18.3.1",
    "react-hook-form": "^7.53.0",
    "tailwind-merge": "^2.5.4",
    "tailwindcss-animate": "^1.0.7",
    "zod": "^3.23.8"
  },
  "devDependencies": {
    "@types/node": "^20",
    "@types/react": "^18",
    "@types/react-dom": "^18",
    "eslint": "^8",
    "eslint-config-next": "15.0.0",
    "postcss": "^8",
    "prisma": "^5.20.0",
    "tailwindcss": "^3.4.1",
    "typescript": "^5"
  }
}
\end{lstlisting}

\subsection{Repository Information}

\begin{table}[H]
\centering
\caption{Informasi Repository}
\begin{tabular}{|l|l|}
\hline
\textbf{Atribut} & \textbf{Value} \\
\hline
Repository Name & scheduler-ai \\
Primary Language & TypeScript \\
Framework & Next.js 15 \\
Database & PostgreSQL + Prisma \\
Deployment & Vercel (Production Ready) \\
Version Control & Git \\
Code Style & ESLint + Prettier \\
Testing & Manual Testing + Performance Testing \\
Documentation & Comprehensive README \\
License & Private/Academic Use \\
\hline
\end{tabular}
\end{table}

\subsection{Development Commands}

\begin{lstlisting}[language=bash, caption=Available Development Commands]
# Development server
npm run dev                 # Start development with Turbopack

# Production build
npm run build              # Build for production
npm run start              # Start production server

# Code quality
npm run lint               # Run ESLint

# Database operations
npx prisma generate        # Generate Prisma client
npx prisma migrate dev     # Run database migrations
npx prisma db seed         # Seed database with initial data
npx prisma studio          # Open database browser

# Package management
npm install                # Install dependencies
npm update                 # Update packages
npm audit                  # Security audit
\end{lstlisting}

\subsection{File Structure Detail}

Source code aplikasi terdiri dari 45+ file TypeScript/JavaScript dengan total sekitar 3,500 lines of code, mencakup:

\begin{itemize}
\item \textbf{API Routes}: 15 endpoint API untuk data management
\item \textbf{React Components}: 25+ reusable components
\item \textbf{Custom Hooks}: 5 custom hooks untuk state management
\item \textbf{Utility Functions}: 10+ helper functions
\item \textbf{Database Models}: 5 Prisma models dengan relationships
\item \textbf{Authentication}: Complete NextAuth.js integration
\item \textbf{AI Integration}: OpenAI dan Claude API integration
\item \textbf{Styling}: Tailwind CSS dengan custom design system
\end{itemize}