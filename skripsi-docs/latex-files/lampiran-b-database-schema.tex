\section{Database Schema}

\subsection{Skema Database Prisma}

Sistem Scheduler AI menggunakan PostgreSQL sebagai database utama dengan Prisma ORM sebagai layer abstraksi data. Berikut adalah struktur lengkap skema database yang digunakan dalam aplikasi:

\begin{lstlisting}[language=JavaScript, caption=Konfigurasi Skema Prisma]
generator client {
  provider = "prisma-client-js"
}

datasource db {
  provider  = "postgresql"
  url       = env("DATABASE_URL")
  directUrl = env("DIRECT_URL")
}
\end{lstlisting}

\subsubsection{Model User}
Model ini menyimpan informasi pengguna dan terintegrasi dengan sistem autentikasi NextAuth.js.

\begin{lstlisting}[language=JavaScript, caption=Skema Model User]
model User {
  id            String     @id @default(cuid())
  name          String?    @db.VarChar(100)
  email         String     @unique
  emailVerified DateTime?
  preferences   Json?
  image         String?
  accounts      Account[]
  sessions      Session[]
  goals         Goal[]
  schedules     Schedule[]

  createdAt DateTime @default(now())
  updatedAt DateTime @updatedAt
}
\end{lstlisting}

\textbf{Deskripsi Field:}
\begin{itemize}
\item \texttt{id}: Primary key menggunakan CUID untuk keamanan
\item \texttt{name}: Nama pengguna (opsional, maksimal 100 karakter)
\item \texttt{email}: Email unik untuk autentikasi
\item \texttt{emailVerified}: Timestamp verifikasi email
\item \texttt{preferences}: Data preferensi pengguna dalam format JSON
\item \texttt{image}: URL avatar pengguna
\item \texttt{accounts}, \texttt{sessions}: Relasi dengan sistem autentikasi
\item \texttt{goals}, \texttt{schedules}: Relasi dengan data aplikasi
\end{itemize}

\subsubsection{Model Goal}
Model ini menyimpan informasi tujuan/goal yang dibuat oleh pengguna dengan sistem tracking progress.

\begin{lstlisting}[language=JavaScript, caption=Skema Model Goal]
model Goal {
  id          String     @id @default(uuid())
  userId      String
  title       String     @db.VarChar(100)
  description String     @db.VarChar(500)
  startDate   DateTime   @default(now())
  endDate     DateTime
  emoji       String     @db.VarChar(20)
  status      StatusGoal
  createdAt   DateTime   @default(now())
  updatedAt   DateTime   @updatedAt
  user        User       @relation(fields: [userId], references: [id])
  schedules   Schedule[]
}

enum StatusGoal {
  ACTIVE
  COMPLETED
  ABANDONED
}
\end{lstlisting}

\textbf{Deskripsi Field:}
\begin{itemize}
\item \texttt{id}: Primary key UUID
\item \texttt{userId}: Foreign key ke tabel User
\item \texttt{title}: Judul goal (maksimal 100 karakter)
\item \texttt{description}: Deskripsi detail goal (maksimal 500 karakter)
\item \texttt{startDate}: Tanggal mulai goal
\item \texttt{endDate}: Target tanggal selesai goal
\item \texttt{emoji}: Emoji representasi goal (maksimal 20 karakter untuk mendukung emoji kompleks)
\item \texttt{status}: Status goal (ACTIVE, COMPLETED, ABANDONED)
\end{itemize}

\subsubsection{Model Schedule}
Model ini menyimpan aktivitas terjadwal yang dikaitkan dengan goal tertentu, termasuk waktu dan status penyelesaian.

\begin{lstlisting}[language=JavaScript, caption=Skema Model Schedule]
model Schedule {
  id              String         @id @default(uuid())
  userId          String
  goalId          String?
  order           String?        @db.VarChar(100)
  title           String         @db.VarChar(100)
  description     String         @db.VarChar(500)
  notes           String?        @db.VarChar(500)
  startedTime     DateTime
  endTime         DateTime
  percentComplete String?        @db.VarChar(3)
  emoji           String         @db.VarChar(20)
  status          StatusSchedule
  createdAt       DateTime       @default(now())
  updatedAt       DateTime       @updatedAt
  goal            Goal?          @relation(fields: [goalId], references: [id])
  user            User           @relation(fields: [userId], references: [id])
}

enum StatusSchedule {
  NONE
  IN_PROGRESS
  COMPLETED
  MISSED
  ABANDONED
}
\end{lstlisting}

\textbf{Deskripsi Field:}
\begin{itemize}
\item \texttt{id}: Primary key UUID
\item \texttt{userId}: Foreign key ke tabel User
\item \texttt{goalId}: Foreign key ke tabel Goal (opsional)
\item \texttt{order}: Urutan aktivitas dalam goal
\item \texttt{title}: Judul aktivitas (maksimal 100 karakter)
\item \texttt{description}: Deskripsi aktivitas (maksimal 500 karakter)
\item \texttt{notes}: Catatan tambahan (opsional, maksimal 500 karakter)
\item \texttt{startedTime}: Waktu mulai aktivitas
\item \texttt{endTime}: Waktu selesai aktivitas
\item \texttt{percentComplete}: Persentase penyelesaian (0-100)
\item \texttt{emoji}: Emoji representasi aktivitas
\item \texttt{status}: Status aktivitas (NONE, IN\_PROGRESS, COMPLETED, MISSED, ABANDONED)
\end{itemize}

\subsubsection{Model Autentikasi NextAuth.js}

\begin{lstlisting}[language=JavaScript, caption=Model Autentikasi NextAuth.js]
model Account {
  userId            String
  type              String
  provider          String
  providerAccountId String
  refresh_token     String?
  access_token      String?
  expires_at        Int?
  token_type        String?
  scope             String?
  id_token          String?
  session_state     String?

  createdAt DateTime @default(now())
  updatedAt DateTime @updatedAt

  user User @relation(fields: [userId], references: [id], onDelete: Cascade)

  @@id([provider, providerAccountId])
}

model Session {
  sessionToken String   @unique
  userId       String
  expires      DateTime
  user         User     @relation(fields: [userId], references: [id], onDelete: Cascade)

  createdAt DateTime @default(now())
  updatedAt DateTime @updatedAt
}

model VerificationToken {
  identifier String
  token      String
  expires    DateTime

  @@id([identifier, token])
}
\end{lstlisting}

\subsection{Relasi Database}

\subsubsection{Diagram Relasi Entitas}
Sistem database memiliki struktur relasi antar model sebagai berikut:

\begin{itemize}
\item \textbf{User} $\leftrightarrow$ \textbf{Goal}: One-to-Many (Satu user dapat memiliki banyak goal)
\item \textbf{User} $\leftrightarrow$ \textbf{Schedule}: One-to-Many (Satu user dapat memiliki banyak schedule)
\item \textbf{Goal} $\leftrightarrow$ \textbf{Schedule}: One-to-Many (Satu goal dapat memiliki banyak schedule)
\item \textbf{User} $\leftrightarrow$ \textbf{Account}: One-to-Many (Satu user dapat memiliki banyak provider account)
\item \textbf{User} $\leftrightarrow$ \textbf{Session}: One-to-Many (Satu user dapat memiliki banyak session)
\end{itemize}

\subsubsection{Constraints dan Validasi}

\begin{itemize}
\item \textbf{Unique Constraints}: Email user harus unik
\item \textbf{Foreign Key Constraints}: Semua relasi menggunakan cascade delete untuk data integrity
\item \textbf{Data Validation}: Panjang maksimal field sudah ditentukan untuk mencegah overflow
\item \textbf{Emoji Support}: Field emoji menggunakan VARCHAR(20) untuk mendukung emoji kompleks seperti 🚴‍♂️
\end{itemize}

\subsection{Strategi Migrasi Database}

Sistem menggunakan Prisma Migrate untuk pengelolaan dan versioning skema database. Berikut adalah perintah-perintah yang digunakan:

\begin{lstlisting}[language=bash, caption=Prisma Migration Commands]
# Generate Prisma client
npx prisma generate

# Create and apply migration
npx prisma migrate dev

# Apply migrations to production
npx prisma migrate deploy

# Seed database with initial data
npx prisma db seed

# Open database browser
npx prisma studio
\end{lstlisting}

\subsection{Layer Validasi Data}

Sistem mengimplementasikan layer validasi data di tingkat aplikasi untuk mencegah pelanggaran constraint database dan memastikan integritas data:

\begin{lstlisting}[language=JavaScript, caption=Data Validation Functions]
// Field validation functions in app/lib/validation.ts
const validateGoalTitle = (title) => title?.slice(0, 100);
const validateGoalDescription = (desc) => desc?.slice(0, 500);
const validateEmoji = (emoji) => emoji?.slice(0, 20);
const validateScheduleNotes = (notes) => notes?.slice(0, 500);
\end{lstlisting}

Fungsi validasi ini memastikan bahwa semua data yang masuk ke database sesuai dengan constraint dan batasan yang telah ditentukan dalam skema, mencegah error runtime dan menjaga konsistensi data.