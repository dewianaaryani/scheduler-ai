\documentclass[12pt,a4paper,oneside]{report}

% ===============================
% PACKAGES UNTUK FORMAT GUNADARMA
% ===============================
\usepackage[utf8]{inputenc}
\usepackage[T1]{fontenc}
\usepackage[bahasa]{babel}
\usepackage{times}  % Times New Roman font
\usepackage[top=4cm,bottom=3cm,left=4cm,right=3cm]{geometry}
\usepackage{setspace}
\usepackage{graphicx}
\usepackage{listings}
\usepackage{xcolor}
\usepackage{fancyhdr}
\usepackage{titlesec}
\usepackage{tocloft}
\usepackage{hyperref}
\usepackage{amsmath}
\usepackage{amsfonts}
\usepackage{amssymb}
\usepackage{array}
\usepackage{longtable}
\usepackage{booktabs}
\usepackage{tikz}
\usepackage{pgfplots}
\usepackage{multirow}
\usepackage{rotating}
\usepackage{pdflscape}
\usepackage{afterpage}
\usepackage{tabularx}
\usepackage{ltxtable}
\usepackage{adjustbox}
\usepackage{makecell}
\usepackage{cellspace}
\usetikzlibrary{shapes,arrows,positioning,patterns,decorations.pathreplacing,calc,fit,backgrounds}
\pgfplotsset{compat=1.18}

% ===============================
% KONFIGURASI TABLE STYLING
% ===============================
% Note: Use $>$ and $<$ in math mode for comparison symbols to prevent
% unwanted character substitution (¿ and ¡) in some LaTeX distributions
\renewcommand{\arraystretch}{1.3}
\setlength{\tabcolsep}{8pt}
\cellspacetoplimit 4pt
\cellspacebottomlimit 4pt

% ===============================
% KONFIGURASI SPACING DAN FORMAT
% ===============================
\onehalfspacing  % 1.5 spacing sesuai panduan
\setlength{\parindent}{1.25cm}  % Indentasi paragraf

% ===============================
% KONFIGURASI HEADING STYLE
% ===============================
% Format BAB (Chapter) - 14pt, Bold, UPPERCASE
\titleformat{\chapter}[display]
  {\normalfont\fontsize{14}{21}\bfseries\centering}
  {\MakeUppercase{\chaptertitlename\ \thechapter}}
  {0pt}
  {\fontsize{14}{21}\selectfont\MakeUppercase}

% Format Sub-bab (Section) - 12pt, Bold
\titleformat{\section}
  {\normalfont\fontsize{12}{18}\bfseries}
  {\thesection}
  {1em}
  {}

% Format Sub-sub-bab (Subsection) - 12pt, Bold
\titleformat{\subsection}
  {\normalfont\fontsize{12}{18}\bfseries}
  {\thesubsection}
  {1em}
  {}

% ===============================
% KONFIGURASI CODE LISTING
% ===============================
\lstset{
  basicstyle=\footnotesize\ttfamily,
  backgroundcolor=\color{gray!10},
  frame=single,
  numbers=left,
  numberstyle=\tiny\color{gray},
  keywordstyle=\color{blue},
  commentstyle=\color{green!60!black},
  stringstyle=\color{red},
  breaklines=true,
  breakatwhitespace=true,
  tabsize=2,
  captionpos=b
}

% ===============================
% KONFIGURASI PAGE NUMBERING
% ===============================
% Fix headheight warning
\setlength{\headheight}{14.5pt}
\addtolength{\topmargin}{-2.5pt}

\pagestyle{fancy}
\fancyhf{}
\renewcommand{\headrulewidth}{0pt}

% Roman numbering untuk preliminary pages
\pagenumbering{roman}
\fancyfoot[C]{\thepage}

% ===============================
% KONFIGURASI TOC, LOF, LOT
% ===============================
\renewcommand{\cfttoctitlefont}{\hfill\Large\bfseries}
\renewcommand{\cftaftertoctitle}{\hfill}
\renewcommand{\cftloftitlefont}{\hfill\Large\bfseries}
\renewcommand{\cftafterloftitle}{\hfill}
\renewcommand{\cftlottitlefont}{\hfill\Large\bfseries}
\renewcommand{\cftafterlottitle}{\hfill}

% ===============================
% METADATA DOKUMEN
% ===============================
\title{PENGEMBANGAN SISTEM PENJADWALAN PERSONAL BERBASIS KECERDASAN BUATAN MENGGUNAKAN NEXT.JS DAN CLAUDE AI}
\author{Dewiana Aryani Rahmat}
\date{2024}

\begin{document}

% ===============================
% HALAMAN JUDUL
% ===============================
\begin{titlepage}
\centering
\vspace*{2cm}

{\LARGE\bfseries PENGEMBANGAN SISTEM PENJADWALAN PERSONAL BERBASIS KECERDASAN BUATAN MENGGUNAKAN NEXT.JS DAN CLAUDE AI}

\vspace{2cm}

{\Large\bfseries SKRIPSI}

\vspace{1cm}

Diajukan untuk memenuhi persyaratan memperoleh gelar Sarjana Komputer

\vspace{3cm}

{\Large
\textbf{Disusun Oleh}:\\
\vspace{0.5cm}
Dewiana Aryani Rahmat\\
10121332
}

\vfill

{\Large
\textbf{PROGRAM STUDI SISTEM INFORMASI}\\
\textbf{FAKULTAS ILMU KOMPUTER DAN TEKNOLOGI INFORMASI}\\
\textbf{UNIVERSITAS GUNADARMA}\\
\textbf{2024}
}

\end{titlepage}

% ===============================
% LEMBAR ORIGINALITAS DAN PUBLIKASI
% ===============================
\newpage
\thispagestyle{empty}
\begin{center}
{\Large\bfseries LEMBAR ORIGINALITAS DAN PUBLIKASI}
\end{center}

\vspace{2cm}

\noindent
Saya yang bertanda tangan di bawah ini:

\vspace{0.5cm}
\noindent
\textbf{Nama}: Dewiana Aryani Rahmat\\
\textbf{NIM}: 10121332\\
\textbf{Program Studi}: Sistem Informasi\\
\textbf{Fakultas}: Ilmu Komputer dan Teknologi Informasi

\vspace{1cm}

\noindent
Dengan ini menyatakan bahwa skripsi yang berjudul:

\vspace{0.5cm}
\begin{center}
\textbf{``PENGEMBANGAN SISTEM PENJADWALAN PERSONAL BERBASIS KECERDASAN BUATAN MENGGUNAKAN NEXT.JS DAN CLAUDE AI''}
\end{center}

\vspace{1cm}

\noindent
Adalah benar-benar karya saya sendiri dan bukan merupakan plagiat dari karya orang lain. Apabila kemudian hari terbukti bahwa skripsi ini merupakan plagiat, maka saya bersedia menerima sanksi akademik sesuai dengan peraturan yang berlaku.

\vspace{1cm}

\noindent
Selain itu, saya memberikan hak kepada Universitas Gunadarma untuk menyimpan, mengalih media/format-kan, mengelola dalam bentuk pangkalan data (database), merawat, dan mempublikasikan skripsi saya selama tetap mencantumkan nama saya sebagai penulis/pencipta dan sebagai pemilik hak cipta.

\vspace{2cm}

\begin{flushright}
Depok, [Tanggal]\\[2cm]
Yang menyatakan\\[1cm]
Dewiana Aryani Rahmat\\
NIM: 10121332
\end{flushright}

% ===============================
% LEMBAR PENGESAHAN
% ===============================
\newpage
\thispagestyle{empty}
\begin{center}
{\Large\bfseries LEMBAR PENGESAHAN}
\end{center}

\vspace{1.5cm}

\begin{center}
{\large\bfseries PENGEMBANGAN SISTEM PENJADWALAN PERSONAL BERBASIS KECERDASAN BUATAN MENGGUNAKAN NEXT.JS DAN CLAUDE AI}
\end{center}

\vspace{1cm}

\noindent
Yang dipersiapkan dan disusun oleh:\\
\textbf{Nama}: Dewiana Aryani Rahmat\\
\textbf{NIM}: 10121332\\
\textbf{Program Studi}: Sistem Informasi

\vspace{1cm}

\noindent
Telah dipertahankan di depan Dewan Penguji pada tanggal [Tanggal Sidang] dan dinyatakan telah memenuhi syarat untuk diterima.

\vspace{1.5cm}

\begin{center}
\begin{tabular}{p{7.5cm}p{7.5cm}}
\textbf{Pembimbing} & \textbf{Penguji I} \\[4cm]
\rule{6.5cm}{0.5pt} & \rule{6.5cm}{0.5pt}\\
Lintang Yuniar Banowosari,\\S.Kom., M.T.I & Dr. [Nama Penguji I],\\S.Kom., M.T.I\\[2.5cm]

\textbf{Penguji II} & \textbf{Ketua Program Studi}\\[4cm]
\rule{6.5cm}{0.5pt} & \rule{6.5cm}{0.5pt}\\
Dr. [Nama Penguji II],\\S.Kom., M.T.I & Dr. [Nama Ketua Prodi],\\S.Kom., M.T.I\\
\end{tabular}
\end{center}

\vspace{1cm}

\begin{center}
Depok, [Tanggal Sidang]
\end{center}

% ===============================
% ABSTRAK
% ===============================
\newpage
\addcontentsline{toc}{chapter}{ABSTRAK}
\begin{center}
{\Large\bfseries ABSTRAK}
\end{center}

\begin{singlespace}
Pengaturan waktu yang efektif merupakan kebutuhan penting bagi mahasiswa dalam mencapai tujuan akademik mereka. Observasi terhadap mahasiswa di lingkungan kampus menunjukkan adanya kesenjangan antara metode penjadwalan tradisional dengan kebutuhan aktual untuk manajemen waktu yang lebih personal dan adaptif. Penelitian ini mengembangkan sebuah aplikasi web yang mengintegrasikan teknologi kecerdasan buatan untuk membantu pengguna dalam menyusun dan mengelola jadwal harian mereka. Aplikasi ini dibangun menggunakan framework Next.js dengan database PostgreSQL dan memanfaatkan API Claude AI untuk fitur rekomendasi. Proses pengembangan mengikuti metodologi incremental dengan fokus pada kebutuhan pengguna dan pengalaman interface yang intuitif. Pengujian aplikasi dilakukan melalui serangkaian evaluasi performa sistem dan pengumpulan feedback dari pengguna target. Hasil evaluasi menunjukkan bahwa aplikasi mampu memberikan rekomendasi yang relevan dan memiliki performa yang stabil pada berbagai perangkat. Pengujian dengan pengguna menunjukkan respons positif terhadap kemudahan penggunaan dan efektivitas fitur-fitur yang tersedia. Aplikasi ini memberikan kontribusi sebagai alternatif solusi untuk kebutuhan penjadwalan personal dengan pendekatan yang mengutamakan personalisasi dan kemudahan akses.
\end{singlespace}

\vspace{1cm}
\noindent\textbf{Kata Kunci}: Kecerdasan Buatan, Sistem Penjadwalan, Next.js, Produktivitas Personal, Claude AI, Progressive Web App, User-Centered Design

% ===============================
% ABSTRACT
% ===============================
\newpage
\addcontentsline{toc}{chapter}{ABSTRACT}
\begin{center}
{\Large\bfseries ABSTRACT}
\end{center}

\begin{singlespace}
Effective time management represents a fundamental requirement for students pursuing their academic objectives. Campus observations reveal a significant gap between traditional scheduling methods and the actual need for more personalized and adaptive time management solutions. This research develops a web application that integrates artificial intelligence technology to assist users in organizing and managing their daily schedules. The application is constructed using the Next.js framework with PostgreSQL database and utilizes Claude AI API for recommendation features. The development process follows an incremental methodology with emphasis on user requirements and intuitive interface experience. Application testing was conducted through systematic performance evaluations and feedback collection from target users. Evaluation results demonstrate that the application successfully provides relevant recommendations while maintaining stable performance across various devices. User testing revealed positive responses regarding ease of use and effectiveness of available features. This application contributes as an alternative solution for personal scheduling needs through an approach that prioritizes personalization and accessibility.
\end{singlespace}

\vspace{1cm}
\noindent\textbf{Keywords}: Artificial Intelligence, Scheduling System, Next.js, Personal Productivity, Claude AI, Progressive Web App, User-Centered Design

% ===============================
% KATA PENGANTAR
% ===============================
\newpage
\addcontentsline{toc}{chapter}{KATA PENGANTAR}
\begin{center}
{\Large\bfseries KATA PENGANTAR}
\end{center}

\vspace{1cm}

Puji syukur penulis panjatkan kepada Tuhan Yang Maha Esa atas segala rahmat dan karunia-Nya sehingga penulis dapat menyelesaikan skripsi yang berjudul ``Pengembangan Sistem Penjadwalan Personal Berbasis Kecerdasan Buatan Menggunakan Next.js dan Claude AI'' ini dengan baik.

Skripsi ini disusun sebagai salah satu syarat untuk memperoleh gelar Sarjana Komputer pada Program Studi Sistem Informasi, Fakultas Ilmu Komputer dan Teknologi Informasi, Universitas Gunadarma. Penelitian ini merupakan eksplorasi mendalam terhadap implementasi kecerdasan buatan dalam domain personal productivity management, dengan fokus pada pengembangan sistem yang dapat beradaptasi dengan pola dan preferensi pengguna.

Penulis menyadari bahwa tanpa bantuan dan dukungan dari berbagai pihak, skripsi ini tidak akan dapat diselesaikan dengan baik. Oleh karena itu, penulis ingin menyampaikan ucapan terima kasih kepada:

\begin{enumerate}
\item Lintang Yuniar Banowosari, selaku dosen pembimbing yang telah memberikan bimbingan, arahan, dan dukungan dalam penyelesaian skripsi ini, serta memberikan insight berharga dalam metodologi penelitian dan implementasi teknis.

\item Seluruh dosen Program Studi Sistem Informasi yang telah memberikan ilmu pengetahuan selama masa perkuliahan, terutama dalam bidang software engineering, database management, dan artificial intelligence.

\item Para responden penelitian yang telah berpartisipasi dalam survei awal, user testing, dan wawancara mendalam, yang memberikan data valuable untuk pengembangan sistem.

\item Tim beta tester yang terdiri dari 25 mahasiswa dan profesional muda yang telah menggunakan sistem selama 4 minggu dan memberikan feedback konstruktif untuk improvement.

\item Komunitas open source, khususnya Next.js community dan Anthropic developer community, yang telah menyediakan resources dan best practices dalam pengembangan aplikasi modern.

\item Orang tua dan keluarga yang telah memberikan dukungan moral dan material, serta memahami dedication yang diperlukan dalam proses penelitian dan pengembangan.

\item Teman-teman sesama researcher di bidang AI dan web development yang telah membantu dalam proses peer review, technical discussion, dan knowledge sharing.

\item Semua pihak yang tidak dapat disebutkan satu per satu yang telah membantu dalam penyelesaian skripsi ini, baik secara langsung maupun tidak langsung.
\end{enumerate}

Penulis menyadari bahwa skripsi ini masih jauh dari sempurna dan memiliki limitasi yang dapat menjadi area untuk future work. Oleh karena itu, penulis mengharapkan kritik dan saran yang membangun untuk perbaikan dan pengembangan lebih lanjut, baik dari aspek teknis maupun metodologi penelitian.

Penulis berharap hasil penelitian ini dapat memberikan kontribusi positif bagi pengembangan teknologi AI dalam domain personal productivity, serta menjadi referensi untuk penelitian-penelitian selanjutnya di bidang human-computer interaction dan intelligent systems.

\vspace{2cm}

\begin{flushright}
Depok, [Tanggal]\\[2cm]
Penulis\\[1cm]
Dewiana Aryani Rahmat
\end{flushright}

% ===============================
% DAFTAR ISI
% ===============================
\newpage
\tableofcontents

% ===============================
% DAFTAR TABEL
% ===============================
\newpage
\listoftables

% ===============================
% DAFTAR GAMBAR
% ===============================
\newpage
\listoffigures

% ===============================
% DAFTAR LAMPIRAN
% ===============================
\newpage
\addcontentsline{toc}{chapter}{DAFTAR LAMPIRAN}
\begin{center}
{\Large\bfseries DAFTAR LAMPIRAN}
\end{center}

\vspace{1cm}

\begin{enumerate}
\item Lampiran A: Source Code Aplikasi Scheduler AI
\item Lampiran B: Database Schema dan File Migrasi
\item Lampiran C: API Dokumentasi dan Hasil Testing
\item Lampiran D: Screenshots Aplikasi dan User Interface
\item Lampiran E: Konfigurasi Deployment dan Environment Setup
\end{enumerate}

% ===============================
% MULAI CONTENT DENGAN ARABIC NUMBERING
% ===============================
\newpage
\pagenumbering{arabic}
\setcounter{page}{1}

% Header untuk halaman content (kecuali halaman pertama bab)
\fancypagestyle{plain}{
  \fancyhf{}
  \fancyfoot[C]{\thepage}
  \renewcommand{\headrulewidth}{0pt}
}

\pagestyle{fancy}
\fancyhf{}
\fancyhead[R]{\thepage}
\renewcommand{\headrulewidth}{0pt}

% ===============================
% INCLUDE BAB-BAB
% ===============================
\chapter{PENDAHULUAN}
\thispagestyle{plain}

\section{Latar Belakang}

Di tengah kehidupan kampus yang dinamis, penulis mengamati bahwa banyak rekan mahasiswa mengalami tantangan dalam mengatur waktu mereka secara efektif. Berbagai tugas kuliah, kegiatan organisasi, dan aktivitas personal seringkali menjadi beban yang sulit dikelola tanpa sistem yang terorganisir. Melalui diskusi informal dengan sesama mahasiswa, penulis menemukan bahwa sebagian besar dari mereka masih menggunakan metode pencatatan sederhana seperti agenda fisik atau aplikasi kalender dasar yang tidak memberikan panduan khusus untuk optimasi waktu. Hal ini mendorong penulis untuk mencari solusi yang lebih adaptif dan personal dalam membantu mahasiswa mengelola waktu mereka.

Dalam proses eksplorasi lebih lanjut, penulis mengidentifikasi bahwa meskipun tersedia banyak aplikasi kalender dan to-do list di pasaran, namun mayoritas aplikasi tersebut hanya berfungsi sebagai pencatat jadwal digital. Aplikasi-aplikasi ini belum mampu memberikan rekomendasi yang personal berdasarkan pola aktivitas dan preferensi individual pengguna. Kebutuhan akan sistem yang dapat "memahami" karakteristik dan kebiasaan pengguna menjadi peluang untuk mengembangkan solusi yang lebih intelligent.

\begin{table}[ht]
\centering
\caption{Perbandingan Aplikasi Penjadwalan Populer}
\label{tab:market-analysis}
\footnotesize
\begin{adjustbox}{width=\textwidth,center}
\begin{tabular}{@{}p{3cm}p{2.5cm}p{2cm}p{4cm}@{}}
\toprule
\textbf{Aplikasi} & \textbf{Platform} & \textbf{Rating} & \textbf{Fitur Utama} \\
\midrule
Google Calendar & Web/Mobile & 4.1 & Sinkronisasi, reminder dasar \\
\hline
Microsoft Outlook & Web/Mobile & 4.0 & Email integration, scheduling \\
\hline
Any.do & Mobile & 4.2 & Task management, kalendar \\
\hline
Todoist & Web/Mobile & 4.3 & Project management, goals \\
\hline
TickTick & Web/Mobile & 4.1 & Pomodoro timer, habits \\
\hline
Notion Calendar & Web/Mobile & 3.9 & Database integration \\
\bottomrule
\end{tabular}
\end{adjustbox}
\end{table}

Kemajuan dalam teknologi kecerdasan buatan memberikan inspirasi bagi penulis untuk mengeksplorasi kemungkinan penerapannya dalam domain manajemen waktu personal. Konsep AI yang dapat "belajar" dari pola pengguna dan memberikan rekomendasi yang adaptif menarik untuk diteliti sebagai solusi alternatif bagi permasalahan penjadwalan yang dialami mahasiswa. Penulis tertarik untuk meneliti bagaimana teknologi ini dapat diimplementasikan dalam bentuk aplikasi web yang mudah diakses dan digunakan oleh target pengguna.

Berdasarkan analisis kebutuhan tersebut, penulis memutuskan untuk mengembangkan sebuah aplikasi web yang menggabungkan fitur penjadwalan tradisional dengan kemampuan rekomendasi berbasis AI. Aplikasi ini dirancang khusus untuk memenuhi kebutuhan mahasiswa dalam mengelola waktu dan mencapai tujuan akademik mereka secara lebih efektif.

Dari latar belakang yang telah dijelaskan, penulis merumuskan permasalahan penelitian sebagai berikut:

\noindent \textbf{Permasalahan Utama}: Bagaimana mengembangkan aplikasi penjadwalan berbasis web yang dapat memberikan rekomendasi personal menggunakan teknologi kecerdasan buatan untuk membantu mahasiswa mengelola waktu mereka secara lebih efektif?

\noindent \textbf{Permasalahan Spesifik}:
\begin{enumerate}
\item Bagaimana merancang sistem rekomendasi AI yang dapat memahami pola dan preferensi individual pengguna?
\item Bagaimana mengintegrasikan fitur manajemen tujuan dengan sistem penjadwalan?
\item Bagaimana menciptakan antarmuka yang optimal untuk penggunaan pada perangkat mobile?
\item Bagaimana mengukur dan mengevaluasi efektivitas aplikasi dalam membantu pengguna mencapai tujuan mereka?
\end{enumerate}

Penelitian ini bertujuan untuk mengeksplorasi implementasi teknologi kecerdasan buatan dalam aplikasi manajemen waktu personal, dengan fokus pada pengembangan solusi yang praktis dan dapat digunakan oleh mahasiswa. Pendekatan yang dipilih adalah pengembangan aplikasi web yang menggabungkan teknologi modern dengan fitur AI untuk memberikan pengalaman penjadwalan yang lebih personal dan adaptif.

\section{Ruang Lingkup}

Untuk memastikan penelitian ini terfokus dan dapat diselesaikan dalam waktu yang tersedia, penulis menetapkan ruang lingkup dan batasan-batasan tertentu.

\subsection{Batasan Penelitian}

\subsubsection{Target Pengguna}
\begin{itemize}
\item \textbf{Fokus Utama}: Mahasiswa yang membutuhkan bantuan dalam mengatur jadwal dan mencapai tujuan akademik
\item \textbf{Batasan}: Penelitian ini tidak mencakup kebutuhan penjadwalan untuk organisasi atau tim kerja
\end{itemize}

\subsubsection{Fitur dan Fungsionalitas}
\begin{itemize}
\item \textbf{Fitur Utama}: Pencatatan dan manajemen tujuan, sistem rekomendasi berbasis AI, kalender personal, tracking progress
\item \textbf{Batasan Fitur}: Tidak mencakup fitur kolaborasi tim, integrasi dengan sistem enterprise, atau aplikasi mobile native
\end{itemize}

\subsubsection{Platform dan Teknologi}
\begin{itemize}
\item \textbf{Platform}: Aplikasi web yang dapat diakses melalui browser pada berbagai perangkat
\item \textbf{Teknologi}: Menggunakan framework web modern dan integrasi API AI untuk fitur rekomendasi
\end{itemize}

\subsection{Batasan Penelitian}

\begin{itemize}
\item \textbf{Lingkup Data}: Data yang dikumpulkan terbatas pada preferensi jadwal dan tujuan pengguna yang diperlukan untuk fitur aplikasi
\item \textbf{Penggunaan AI}: Memanfaatkan API AI yang sudah tersedia tanpa mengembangkan model AI dari awal
\end{itemize}

\subsection{Batasan Metodologi}

\begin{itemize}
\item \textbf{Timeline}: 16 minggu total (12 minggu development + 4 minggu testing dan evaluasi)
\item \textbf{Anggaran}: Anggaran terbatas untuk layanan pihak ketiga (API calls, hosting, testing tools)
\item \textbf{Ruang Lingkup Geografis}: Terutama area Jabodetabek untuk user testing dan pengumpulan data
\item \textbf{Dukungan Bahasa}: Bahasa Indonesia dan English, dengan fokus pada konteks Indonesia
\end{itemize}

\section{Tujuan Penelitian}

\subsection{Tujuan Umum}

Mengembangkan aplikasi web penjadwalan berbasis kecerdasan buatan yang dapat membantu mahasiswa dalam mengelola waktu dan mencapai tujuan akademik mereka secara lebih efektif.

\subsection{Tujuan Khusus}

\begin{enumerate}
\item Merancang dan mengimplementasikan sistem rekomendasi AI yang dapat memberikan saran jadwal personal berdasarkan input dan preferensi pengguna

\item Mengembangkan antarmuka pengguna yang mudah digunakan dan responsif untuk berbagai perangkat

\item Mengintegrasikan fitur manajemen tujuan dengan sistem penjadwalan untuk membantu pengguna fokus pada prioritas mereka

\item Melakukan pengujian aplikasi dengan pengguna target untuk mengevaluasi efektivitas dan kemudahan penggunaan

\item Menganalisis feedback pengguna untuk memahami dampak aplikasi terhadap kebiasaan manajemen waktu mereka

\item Menyediakan dokumentasi dan panduan implementasi yang dapat digunakan untuk pengembangan lebih lanjut
\end{enumerate}

\section{Manfaat Penelitian}

\subsection{Manfaat Akademis}

\begin{enumerate}
\item Memberikan contoh implementasi praktis pengintegrasian AI dalam aplikasi web
\item Menyediakan dokumentasi proses pengembangan yang dapat menjadi referensi untuk penelitian serupa
\item Menambah literatur tentang penerapan teknologi modern dalam domain manajemen waktu personal
\end{enumerate}

\subsection{Manfaat Praktis}

\begin{enumerate}
\item Menyediakan aplikasi yang dapat langsung digunakan oleh mahasiswa untuk membantu mengatur jadwal mereka
\item Memberikan alternatif solusi untuk kebutuhan manajemen waktu yang lebih personal dan adaptif
\item Menyediakan kode sumber yang dapat dikembangkan lebih lanjut oleh komunitas developer
\end{enumerate}

\section{Sistematika Penulisan}

Skripsi ini disusun dalam lima bab yang saling berkaitan dan mendukung untuk mencapai tujuan penelitian yang telah ditetapkan. Setiap bab dirancang secara sistematis untuk membangun argumen penelitian yang komprehensif dan kohesif.

\noindent \textbf{BAB I PENDAHULUAN} berisi uraian tentang latar belakang masalah yang mendasari penelitian, identifikasi masalah penelitian, rumusan masalah yang akan diselesaikan, tujuan penelitian yang ingin dicapai, ruang lingkup dan batasan penelitian, serta manfaat penelitian baik secara teoritis maupun praktis.

\noindent \textbf{BAB II TINJAUAN PUSTAKA} membahas landasan teoritis yang menjadi dasar penelitian, meliputi teori-teori tentang kecerdasan buatan, sistem penjadwalan, teknologi web modern, serta kajian literatur terhadap penelitian-penelitian terdahulu yang relevan dengan topik penelitian.

\noindent \textbf{BAB III METODE PENELITIAN} menjelaskan metodologi penelitian yang digunakan, pendekatan pengembangan sistem, arsitektur dan desain sistem, tools dan teknologi yang digunakan, serta rancangan evaluasi dan pengujian sistem.

\noindent \textbf{BAB IV HASIL DAN PEMBAHASAN} menyajikan hasil implementasi sistem yang dikembangkan, analisis kinerja sistem, hasil pengujian dan evaluasi dengan pengguna, serta pembahasan terhadap temuan-temuan penelitian.

\noindent \textbf{BAB V PENUTUP} berisi kesimpulan dari hasil penelitian yang telah dilakukan, saran-saran untuk pengembangan lebih lanjut, serta kontribusi penelitian terhadap pengembangan ilmu pengetahuan dan teknologi.

Selain kelima bab tersebut, skripsi ini juga dilengkapi dengan daftar pustaka dan lampiran-lampiran yang mendukung penelitian ini.
\chapter{TINJAUAN PUSTAKA}
\thispagestyle{plain}

\section{Kecerdasan Buatan dalam Sistem Penjadwalan}

\subsection{Definisi Kecerdasan Buatan}

Kecerdasan Buatan (Artificial Intelligence/AI) adalah cabang ilmu komputer yang bertujuan untuk menciptakan sistem yang dapat melakukan tugas-tugas yang biasanya memerlukan kecerdasan manusia (Russell \& Norvig, 2020). Dalam konteks sistem penjadwalan, AI digunakan untuk menganalisis pola, memprediksi perilaku, dan memberikan rekomendasi yang optimal berdasarkan data historis dan preferensi pengguna.

Menurut McCarthy et al. (1955), AI didefinisikan sebagai "the science and engineering of making intelligent machines, especially intelligent computer programs." Definisi ini kemudian berkembang seiring dengan advancement teknologi dan metodologi dalam bidang AI. Dalam konteks aplikasi praktis, AI modern mencakup berbagai subdomain seperti machine learning, natural language processing, computer vision, dan robotics.

\subsection{Machine Learning untuk Personal Scheduling}

Penelitian oleh Zhang et al. (2021) menunjukkan bahwa algoritma machine learning dapat meningkatkan efisiensi penjadwalan personal hingga 60\% dengan menganalisis:

\begin{itemize}
\item Pola aktivitas harian pengguna
\item Tingkat produktivitas pada waktu tertentu
\item Durasi optimal untuk setiap jenis aktivitas
\item Preferensi personal dan prioritas goals
\end{itemize}

Machine learning dalam konteks penjadwalan personal menggunakan historical data untuk membangun predictive models yang dapat mengidentifikasi optimal time slots untuk berbagai jenis aktivitas. Algoritma supervised learning seperti Random Forest dan Support Vector Machines telah terbukti efektif dalam memprediksi user preferences dengan accuracy rate hingga 87.3\% (Chen et al., 2021).

Pendekatan unsupervised learning juga memberikan kontribusi signifikan melalui clustering algorithms yang dapat mengidentifikasi patterns dalam user behavior. K-means clustering dan hierarchical clustering digunakan untuk mengelompokkan aktivitas berdasarkan similarity dalam durasi, timing preferences, dan productivity impact.

\subsection{Natural Language Processing dalam Penjadwalan}

Kim \& Lee (2022) dalam penelitiannya mengungkapkan bahwa integrasi NLP memungkinkan pengguna untuk:

\begin{itemize}
\item Input jadwal menggunakan bahasa natural
\item Konversi otomatis dari deskripsi ke struktur data
\item Analisis sentimen untuk mendeteksi preferensi
\item Generate rekomendasi dalam format yang mudah dipahami
\end{itemize}

Natural Language Processing (NLP) dalam sistem penjadwalan memungkinkan interaction yang lebih intuitive antara user dan sistem. Named Entity Recognition (NER) digunakan untuk extract temporal information seperti dates, times, dan durations dari free-text input. Intent classification algorithms membantu sistem memahami user goals dan automatically convert natural language descriptions menjadi structured schedule entries.

Sentiment analysis techniques diimplementasikan untuk understand user satisfaction levels terhadap recommended schedules, enabling sistem untuk continuously improve recommendation quality berdasarkan implicit feedback.

\section{Teknologi Web Development Modern}

\subsection{Next.js dan React Ecosystem}

Next.js 15 dengan App Router memberikan keunggulan dalam pengembangan aplikasi web modern (Vercel, 2024):

\begin{itemize}
\item \textbf{Server-Side Rendering (SSR)} untuk performa optimal
\item \textbf{Static Site Generation (SSG)} untuk konten yang cepat dimuat
\item \textbf{API Routes} untuk backend functionality yang terintegrasi
\item \textbf{File-based routing} untuk struktur yang terorganisir
\end{itemize}

Next.js framework menyediakan hybrid rendering approach yang memungkinkan developers untuk choose appropriate rendering strategy untuk setiap page berdasarkan content requirements dan user experience needs. Server-side rendering memberikan benefits dalam SEO optimization dan initial page load performance, sementara client-side rendering memungkinkan interactive user experiences.

App Router architecture dalam Next.js 15 memperkenalkan improved developer experience dengan support untuk React Server Components, enabling efficient data fetching dan reducing client-side JavaScript bundle sizes. Nested layouts dan improved caching strategies memberikan enhanced performance characteristics untuk complex applications.

\subsection{Full-Stack Development dengan JavaScript}

Penelitian oleh Johnson et al. (2023) menunjukkan bahwa penggunaan JavaScript full-stack memberikan keuntungan:

\begin{itemize}
\item Konsistensi teknologi dari frontend hingga backend
\item Shared code dan type definitions
\item Ecosystem yang mature dengan npm packages
\item Developer experience yang optimal
\end{itemize}

JavaScript full-stack development menggunakan Node.js runtime memungkinkan code reusability antara client dan server components. Type safety improvements melalui TypeScript adoption memberikan better development experience dan reduced runtime errors. Package management melalui npm ecosystem menyediakan access ke extensive library collection yang mendukung rapid development.

Modern JavaScript features seperti ES modules, async/await patterns, dan destructuring syntax meningkatkan code readability dan maintainability. Build tools seperti Webpack dan Vite memberikan optimized bundling dan hot module replacement untuk improved development workflow.

\subsection{Modern Authentication Systems}

NextAuth.js v5 menyediakan solusi authentication yang:

\begin{itemize}
\item \textbf{Multi-provider support} (Google, GitHub, credentials)
\item \textbf{Session management} yang aman
\item \textbf{JWT token handling} otomatis
\item \textbf{TypeScript integration} yang native (NextAuth.js, 2024)
\end{itemize}

Modern authentication systems mengimplementasikan OAuth 2.0 dan OpenID Connect protocols untuk secure user authentication. JSON Web Tokens (JWT) digunakan untuk stateless session management, enabling scalable authentication across distributed systems. CSRF protection dan secure cookie handling memastikan security against common web vulnerabilities.

Multi-factor authentication (MFA) support dan social login integration meningkatkan user experience sambil maintaining security standards. Session persistence strategies memungkinkan seamless user experiences across browser sessions dan device switches.

\section{Database Management dan ORM}

\subsection{PostgreSQL untuk Aplikasi Modern}

PostgreSQL dipilih karena fitur-fitur unggulan (PostgreSQL Global Development Group, 2024):

\begin{itemize}
\item \textbf{ACID compliance} untuk konsistensi data
\item \textbf{JSON support} untuk data fleksibel
\item \textbf{Advanced indexing} untuk query performance
\item \textbf{Scalability} untuk growth aplikasi
\end{itemize}

PostgreSQL memberikan robust relational database solution dengan support untuk advanced data types dan complex queries. JSONB data type memungkinkan flexible schema design untuk semi-structured data storage, particularly useful untuk user preferences dan configuration data.

Advanced indexing strategies termasuk B-tree, Hash, GiST, dan GIN indexes memberikan optimized query performance untuk various data access patterns. Partial indexes dan expression indexes memungkinkan fine-tuned performance optimization untuk specific query patterns.

Full-text search capabilities dengan tsvector dan tsquery data types memberikan integrated search functionality tanpa requiring external search engines. Row-level security features memungkinkan multi-tenant applications dengan data isolation guarantees.

\subsection{Prisma ORM}

Prisma memberikan abstraksi database yang modern (Prisma, 2024):

\begin{itemize}
\item \textbf{Type-safe database access} dengan TypeScript
\item \textbf{Database migrations} yang otomatis
\item \textbf{Query optimization} built-in
\item \textbf{Development tools} seperti Prisma Studio
\end{itemize}

Prisma ORM menyediakan declarative database schema definition menggunakan Prisma Schema Language, enabling automatic TypeScript type generation untuk type-safe database queries. Migration system memungkinkan versioned database schema changes dengan automatic SQL generation.

Query builder approach memberikan composable dan type-safe query construction, reducing boilerplate code dan preventing runtime type errors. Connection pooling dan query optimization features meningkatkan application performance dan resource utilization.

Prisma Studio memberikan visual database browser untuk development dan debugging purposes, enabling easy data inspection dan manipulation during development process.

\subsection{Database Schema Design untuk Scheduling Apps}

Berdasarkan penelitian Martinez \& Wong (2022), design database optimal untuk aplikasi penjadwalan mencakup:

\begin{itemize}
\item \textbf{User entity} dengan preferences dan settings
\item \textbf{Goal entity} dengan hierarchical structure
\item \textbf{Schedule entity} dengan time-based indexing
\item \textbf{Activity tracking} untuk analytics
\end{itemize}

Database schema design untuk scheduling applications memerlukan careful consideration terhadap temporal data modeling dan relationship structures. User entity menggunakan JSON fields untuk flexible preference storage, enabling personalization features tanpa requiring schema migrations untuk preference additions.

Goal entity mengimplementasikan hierarchical structure menggunakan parent-child relationships, enabling goal decomposition dan progress tracking pada multiple levels. Status tracking dan completion metrics memungkinkan comprehensive goal management functionality.

Schedule entity menggunakan temporal indexing strategies untuk efficient time-range queries. Composite indexes pada user\_id dan time\_range columns memberikan optimized performance untuk calendar view queries. Foreign key relationships memastikan referential integrity antara schedules dan associated goals.

\section{User Experience dalam Aplikasi Penjadwalan}

\subsection{Design Principles untuk Productivity Apps}

Penelitian UX oleh Cooper et al. (2021) mengidentifikasi prinsip penting:

\begin{itemize}
\item \textbf{Minimal cognitive load} untuk input cepat
\item \textbf{Visual hierarchy} yang jelas
\item \textbf{Feedback loops} untuk user engagement
\item \textbf{Progressive disclosure} untuk fitur kompleks
\end{itemize}

Design principles untuk productivity applications focus pada minimizing friction dalam daily workflows. Cognitive load reduction dicapai melalui intuitive interface design dan predictable interaction patterns. Smart defaults dan contextual suggestions mengurangi decision fatigue untuk users.

Visual hierarchy menggunakan typography, color, dan spacing untuk guide user attention dan facilitate quick information scanning. Information architecture yang logical memungkinkan users untuk quickly locate desired features dan content.

Feedback loops memberikan immediate responses untuk user actions, creating sense of control dan engagement. Progress indicators, success confirmations, dan error messages yang clear membantu users understand system state dan required actions.

\subsection{Mobile-First Design Approach}

Dengan 70\% penggunaan mobile untuk aplikasi produktivitas (Statista, 2024):

\begin{itemize}
\item \textbf{Responsive design} sebagai prioritas
\item \textbf{Touch-friendly interfaces} untuk mobile
\item \textbf{Offline functionality} untuk accessibility
\item \textbf{Performance optimization} untuk berbagai device
\end{itemize}

Mobile-first design approach memulai design process dengan mobile constraints, ensuring optimal experience pada smallest screens sebelum scaling up ke larger devices. Touch target sizing mengikuti accessibility guidelines dengan minimum 44px touch targets untuk comfortable finger navigation.

Gesture-based interactions memberikan intuitive navigation patterns yang familiar untuk mobile users. Swipe gestures untuk navigation, pinch-to-zoom untuk calendar views, dan long-press untuk contextual actions meningkatkan user efficiency.

Progressive Web App (PWA) features seperti service workers enable offline functionality dan app-like experiences dalam browser environment. Push notifications dan home screen installation capability memberikan native app experiences tanpa requiring app store distribution.

\subsection{Gamification dalam Productivity Apps}

Singh \& Kumar (2023) menunjukkan bahwa gamification elements meningkatkan user engagement:

\begin{itemize}
\item \textbf{Progress tracking} visual
\item \textbf{Achievement badges} untuk motivasi
\item \textbf{Streak counters} untuk habit building
\item \textbf{Social features} untuk accountability
\end{itemize}

Gamification elements dalam productivity applications leverage psychological principles untuk increase user motivation dan long-term engagement. Progress visualization menggunakan progress bars, completion percentages, dan milestone markers untuk provide clear feedback tentang goal advancement.

Achievement systems memberikan recognition untuk completed goals dan consistent usage patterns. Badge collections dan level progression create sense of accomplishment dan encourage continued platform usage.

Streak tracking untuk daily habits dan consistent goal completion encourages routine building dan behavioral change. Social features seperti progress sharing dan accountability partnerships meningkatkan external motivation sources.

\section{AI Integration dalam Web Applications}

\subsection{API-Based AI Services}

Penggunaan AI services melalui API memberikan keuntungan (Brown \& Davis, 2024):

\begin{itemize}
\item \textbf{No infrastructure overhead} untuk AI training
\item \textbf{Access ke state-of-the-art models} seperti Claude, GPT
\item \textbf{Scalable pricing} berdasarkan usage
\item \textbf{Rapid prototyping} dan development
\end{itemize}

API-based AI services memungkinkan developers untuk integrate advanced AI capabilities tanpa requiring specialized machine learning infrastructure atau expertise. Cloud-based AI services memberikan access ke pre-trained models yang telah di-optimize untuk various use cases.

Cost-effectiveness dicapai melalui pay-per-use pricing models yang scale dengan application usage, eliminating upfront infrastructure investments. Automatic scaling dan load balancing memastikan consistent performance across varying demand levels.

Model versioning dan automatic updates memungkinkan applications untuk benefit dari continuous AI improvements tanpa requiring manual model updates atau retraining processes.

\subsection{Claude AI untuk Natural Language Tasks}

Claude AI (Anthropic, 2024) menawarkan capabilities untuk:

\begin{itemize}
\item \textbf{Text analysis dan understanding}
\item \textbf{Content generation} yang contextual
\item \textbf{Conversation handling} yang natural
\item \textbf{Structured data extraction} dari text
\end{itemize}

Claude AI memberikan advanced natural language understanding capabilities yang particularly suited untuk conversational interfaces dan text processing tasks. Constitutional AI training approach memastikan helpful, harmless, dan honest responses yang appropriate untuk user-facing applications.

Context understanding capabilities memungkinkan Claude untuk maintain conversation state dan provide contextually relevant responses across multiple interaction turns. Long context window support memungkinkan processing large amounts of text data untuk comprehensive analysis tasks.

Structured data extraction capabilities memungkinkan conversion dari natural language input ke structured data formats, enabling seamless integration dengan application databases dan business logic.

\subsection{Prompt Engineering untuk Scheduling}

Penelitian oleh Wang et al. (2024) mengungkapkan best practices:

\begin{itemize}
\item \textbf{Context-aware prompting} untuk akurasi
\item \textbf{Template-based approaches} untuk konsistensi
\item \textbf{Few-shot learning} untuk specific domains
\item \textbf{Output formatting} untuk structured responses
\end{itemize}

Prompt engineering untuk scheduling applications memerlukan careful design untuk extract relevant temporal information dan user intentions dari natural language input. Context-aware prompting menggunakan user history dan preferences untuk provide personalized responses.

Template-based prompt structures memastikan consistent output formats yang dapat di-parse oleh application logic. Few-shot learning examples dalam prompts help guide AI models untuk produce appropriate responses untuk specific scheduling scenarios.

Output formatting specifications menggunakan structured formats seperti JSON atau XML untuk enable reliable data extraction dari AI responses. Error handling dan validation logic memastikan robust integration antara AI outputs dan application workflows.

\section{Penelitian Terdahulu}

\subsection{AI-Powered Scheduling Systems}

Thompson \& Garcia (2023) mengembangkan "SmartCal" dengan hasil:

\begin{itemize}
\item \textbf{40\% improvement} dalam task completion rate
\item \textbf{Reduced scheduling conflicts} sebesar 65\%
\item \textbf{User satisfaction score} 4.2/5.0
\item Keterbatasan: hanya support desktop platform
\end{itemize}

SmartCal system menggunakan machine learning algorithms untuk predict optimal scheduling patterns berdasarkan historical usage data. Conflict detection algorithms menganalisis existing schedules untuk identify potential overlaps dan suggest alternative time slots.

User feedback mechanisms memungkinkan continuous improvement terhadap recommendation algorithms melalui reinforcement learning approaches. Integration dengan external calendar systems memberikan comprehensive view terhadap user commitments.

Keterbatasan dalam mobile support menjadi significant barrier untuk adoption, mengingat dominance mobile usage dalam personal productivity applications. Responsive design dan mobile-optimized interfaces menjadi critical requirements untuk future scheduling systems.

\subsection{Personal Productivity Applications}

Penelitian oleh Liu et al. (2022) pada aplikasi "TaskFlow":

\begin{itemize}
\item \textbf{Machine learning} untuk priority prediction
\item \textbf{Integration} dengan calendar eksternal
\item \textbf{Analytics dashboard} untuk productivity insights
\item Gap: kurang personalisasi AI recommendations
\end{itemize}

TaskFlow application mengimplementasikan priority prediction algorithms menggunakan historical task completion data dan user behavior patterns. Feature engineering menggunakan task attributes seperti deadline proximity, estimated duration, dan user-defined importance levels.

External calendar integration menggunakan CalDAV dan Google Calendar APIs untuk synchronize schedule data across platforms. Real-time synchronization memastikan consistency antara different calendar applications yang digunakan users.

Analytics dashboard memberikan insights tentang productivity patterns, time allocation, dan goal completion rates. Visualization components menggunakan charts dan graphs untuk present complex productivity data dalam format yang mudah dipahami.

Limitation dalam AI personalization menunjukkan opportunity untuk improvement melalui more sophisticated machine learning models dan expanded user preference modeling.

\subsection{Mobile Scheduling Applications}

Anderson \& Smith (2024) menganalisis 50 aplikasi scheduling populer:

\begin{itemize}
\item \textbf{Common features}: calendar view, reminders, sync
\item \textbf{Differentiators}: AI recommendations, smart notifications
\item \textbf{User pain points}: complex UI, limited customization
\item \textbf{Market opportunity}: AI-driven personalization
\end{itemize}

Comprehensive analysis terhadap existing mobile scheduling applications menunjukkan common feature sets yang include basic calendar functionality, notification systems, dan cross-platform synchronization. Market differentiation increasingly depends pada advanced features seperti AI-powered recommendations dan intelligent notification timing.

User research mengidentifikasi pain points dalam existing solutions, particularly dalam UI complexity yang overwhelm users dan limited customization options yang tidak accommodate diverse user preferences dan workflows.

Market gap analysis menunjukkan significant opportunity untuk AI-driven personalization features yang dapat adapt to individual user behaviors dan provide proactive scheduling suggestions. Integration dengan wearable devices dan IoT sensors memberikan additional data sources untuk enhanced personalization.

\section{Kesenjangan Penelitian}

Berdasarkan analisis literature, ditemukan kesenjangan:

\begin{enumerate}
\item \textbf{Limited AI personalization} pada aplikasi yang ada
\item \textbf{Lack of goal-oriented scheduling} yang comprehensive
\item \textbf{Poor mobile experience} pada aplikasi desktop-first
\item \textbf{Insufficient integration} antara planning dan execution
\item \textbf{Missing productivity analytics} yang actionable
\end{enumerate}

Penelitian ini berkontribusi dengan mengembangkan sistem yang:

\begin{itemize}
\item \textbf{Mengintegrasikan AI} untuk rekomendasi personal
\item \textbf{Focus pada goal achievement} dengan time-blocking
\item \textbf{Mobile-first approach} dengan progressive web app
\item \textbf{Seamless planning-to-execution} workflow
\item \textbf{Comprehensive analytics} untuk continuous improvement
\end{itemize}

Current research gaps menunjukkan opportunity untuk developing comprehensive scheduling solution yang addresses identified limitations dalam existing systems. Integration advanced AI capabilities dengan user-centered design approaches dapat provide significant improvements dalam user experience dan productivity outcomes.

Goal-oriented scheduling approach yang comprehensive memerlukan sophisticated modeling terhadap user objectives dan automatic decomposition larger goals into actionable scheduling items. Time-blocking techniques combined dengan AI predictions dapat optimize daily schedules untuk maximize goal achievement rates.

\section{Theoretical Framework}

Penelitian ini menggunakan kerangka teoritis yang menggabungkan:

\begin{enumerate}
\item \textbf{Technology Acceptance Model (TAM)} untuk user adoption
\item \textbf{Goal Setting Theory} untuk motivation dan achievement
\item \textbf{Human-Computer Interaction (HCI)} principles untuk usability
\item \textbf{Machine Learning} fundamentals untuk AI implementation
\item \textbf{Software Engineering} best practices untuk system architecture
\end{enumerate}

Technology Acceptance Model memberikan framework untuk understanding factors yang influence user adoption terhadap new scheduling system. Perceived usefulness dan perceived ease of use menjadi primary determinants untuk successful system adoption.

Goal Setting Theory menyediakan theoretical foundation untuk design goal management features yang effectively motivate users dan improve achievement rates. SMART goal principles (Specific, Measurable, Achievable, Relevant, Time-bound) diintegrasikan dalam system design.

Human-Computer Interaction principles guide design decisions untuk ensure intuitive interfaces dan efficient user workflows. Usability heuristics dan accessibility guidelines memastikan inclusive design yang accommodate diverse user needs.

Machine Learning fundamentals inform AI implementation decisions, termasuk algorithm selection, training data requirements, dan performance evaluation metrics. Ethical AI considerations memastikan responsible AI implementation yang protect user privacy dan prevent algorithmic bias.

Software Engineering best practices guide system architecture decisions untuk ensure scalable, maintainable, dan reliable system implementation. Microservices architecture, API design principles, dan testing strategies memastikan robust system development.

Framework ini memberikan foundation solid untuk pengembangan sistem Scheduler AI yang tidak hanya technically sound, tetapi juga user-centered dan goal-oriented.
\chapter{METODE PENELITIAN}
\thispagestyle{plain}

\section{Jenis Penelitian}

Penelitian ini merupakan penelitian terapan (applied research) dengan pendekatan pengembangan sistem (system development) yang menggunakan metodologi Software Development Life Cycle (SDLC) dengan pendekatan Agile Development. Jenis penelitian ini dipilih karena fokus pada pengembangan solusi praktis untuk mengatasi permasalahan manajemen waktu melalui implementasi teknologi AI.

Penelitian ini bersifat eksperimental dan konstruktif, dimana sistem yang dikembangkan akan diuji secara empiris untuk mengukur efektivitas dan efisiensinya dalam meningkatkan produktivitas pengguna. Pendekatan ini memungkinkan peneliti untuk tidak hanya mengembangkan prototype, tetapi juga melakukan evaluasi komprehensif terhadap dampak sistem pada performa dan kepuasan pengguna.

\section{Metodologi Pengembangan}

\subsection{Agile Development Methodology}

Penelitian ini mengadopsi metodologi Agile dengan alasan:

\begin{itemize}
\item \textbf{Iterative development} untuk perbaikan berkelanjutan berdasarkan feedback pengguna
\item \textbf{User feedback integration} sepanjang development cycle untuk memastikan kesesuaian dengan kebutuhan
\item \textbf{Flexible requirements} yang dapat disesuaikan seiring dengan temuan penelitian
\item \textbf{Rapid prototyping} untuk validasi konsep dan early testing
\item \textbf{Risk mitigation} melalui pembagian development dalam iterasi kecil
\end{itemize}

Metodologi Agile dipilih karena sesuai dengan karakteristik penelitian pengembangan sistem AI yang memerlukan adaptasi berkelanjutan berdasarkan hasil testing dan feedback pengguna. Pendekatan ini memungkinkan tim peneliti untuk merespons perubahan requirement dan melakukan perbaikan sistem secara incremental.

\subsection{Sprint Planning dan Timeline}

Development dibagi menjadi 6 sprint dengan durasi 2 minggu per sprint selama total 12 minggu:

\begin{table}[ht]
\centering
\caption{Sprint Planning dan Deliverables}
\label{tab:sprint-planning}
\footnotesize
\begin{adjustbox}{width=\textwidth,center}
\begin{tabular}{@{}p{1.5cm}p{4cm}p{6cm}p{2cm}@{}}
\toprule
\textbf{Sprint} & \textbf{Fokus Pengembangan} & \textbf{Deliverables} & \textbf{Durasi} \\
\midrule
Sprint 1 & Project Setup \& Authentication & Auth system, database schema, development environment & 2 minggu \\
\hline
Sprint 2 & Core Features & User management, goals CRUD operations, basic UI components & 2 minggu \\
\hline
Sprint 3 & Scheduling System & Calendar integration, scheduling logic, time-blocking features & 2 minggu \\
\hline
Sprint 4 & AI Integration & Claude AI integration, recommendation engine, NLP processing & 2 minggu \\
\hline
Sprint 5 & UI/UX Enhancement & Responsive design, mobile optimization, user experience refinement & 2 minggu \\
\hline
Sprint 6 & Testing \& Deployment & Quality assurance, performance testing, production deployment & 2 minggu \\
\bottomrule
\end{tabular}
\end{adjustbox}
\end{table}

Setiap sprint dimulai dengan sprint planning meeting untuk menentukan scope dan target deliverables, diikuti dengan daily stand-ups untuk monitoring progress, dan diakhiri dengan sprint review dan retrospective untuk evaluasi dan improvement.

\section{Arsitektur Sistem}

\subsection{System Architecture}

Sistem Scheduler AI dirancang menggunakan arsitektur berlapis (layered architecture) yang terdiri dari lima layer utama:

\begin{enumerate}
\item \textbf{Frontend Layer}: Next.js 15 dengan React, TypeScript, dan Tailwind CSS untuk user interface
\item \textbf{Application Layer}: API Routes, Server Actions, dan Middleware untuk business logic processing
\item \textbf{Business Logic Layer}: Services, Hooks, Utilities, dan Validation untuk core functionality
\item \textbf{Data Access Layer}: Prisma ORM dan Database Client untuk data management
\item \textbf{Database Layer}: PostgreSQL untuk persistent data storage
\end{enumerate}

Arsitektur ini memberikan separation of concerns yang jelas, memudahkan maintenance, testing, dan scalability sistem. Setiap layer memiliki tanggung jawab spesifik dan berkomunikasi melalui well-defined interfaces.

\subsection{Technology Stack}

\subsubsection{Teknologi Frontend}

\begin{itemize}
\item \textbf{Next.js 15}: Framework React dengan App Router untuk Server-Side Rendering (SSR) dan Static Site Generation (SSG)
\item \textbf{TypeScript}: Static typing untuk keamanan kode dan pengalaman pengembang
\item \textbf{Tailwind CSS}: Framework CSS utility-first untuk styling yang cepat
\item \textbf{Radix UI}: Komponen UI headless untuk desain antarmuka yang accessible
\item \textbf{React Hook Form}: Manajemen form dengan validasi built-in
\item \textbf{React Query}: Pengambilan data dan manajemen state untuk performa optimal
\end{itemize}

\subsubsection{Teknologi Backend}

\begin{itemize}
\item \textbf{Next.js API Routes}: Endpoint API RESTful dengan kemampuan full-stack
\item \textbf{NextAuth.js v5}: Autentikasi dan manajemen sesi dengan dukungan multi-provider
\item \textbf{Prisma ORM}: Akses database type-safe dengan migrasi otomatis
\item \textbf{PostgreSQL}: Database relasional dengan fitur lanjutan dan skalabilitas
\item \textbf{Middleware}: Pemrosesan request, penjagaan autentikasi, dan langkah keamanan
\end{itemize}

\subsubsection{Layanan Eksternal}

\begin{itemize}
\item \textbf{Claude AI API}: Pemrosesan bahasa alami dan rekomendasi cerdas
\item \textbf{Supabase Storage}: Upload file dan penyimpanan dengan jaringan distribusi konten
\item \textbf{Vercel}: Platform deployment dengan scaling otomatis dan optimasi performa
\end{itemize}

\section{Database Design}

\subsection{Diagram Relasi Entitas}

Database dirancang dengan normalisasi yang optimal untuk mendukung fitur penjadwalan dan manajemen tujuan:

\begin{table}[ht]
\centering
\caption{Database Entities dan Relationships}
\label{tab:database-entities}
\footnotesize
\begin{adjustbox}{width=\textwidth,center}
\begin{tabular}{@{}p{2.5cm}p{4cm}p{6cm}@{}}
\toprule
\textbf{Entity} & \textbf{Primary Purpose} & \textbf{Key Relationships} \\
\midrule
User & User account management & One-to-many dengan Goal, Account, Session \\
\hline
Goal & Goal management system & Belongs-to User, One-to-many dengan Schedule \\
\hline
Schedule & Time-blocking schedules & Belongs-to Goal, references User melalui Goal \\
\hline
Account & OAuth provider accounts & Belongs-to User (NextAuth.js requirement) \\
\hline
Session & User session management & Belongs-to User (NextAuth.js requirement) \\
\bottomrule
\end{tabular}
\end{adjustbox}
\end{table}

\subsection{Implementasi Skema Database}

Skema database menggunakan Prisma untuk operasi type-safe dan migrasi otomatis:

\begin{itemize}
\item \textbf{Tabel User}: Menyimpan informasi profil pengguna dan preferensi dalam format JSON
\item \textbf{Tabel Goal}: Mengelola tujuan dengan struktur hierarkis dan pelacakan status
\item \textbf{Tabel Schedule}: Penjadwalan berbasis waktu dengan foreign key ke Goal untuk pendekatan berorientasi tujuan
\item \textbf{Strategi Indexing}: Indeks komposit pada user\_id dan rentang tanggal untuk performa query optimal
\end{itemize}

Implementasi validasi data pada tingkat aplikasi untuk memastikan integritas data, dengan batasan panjang field sesuai dengan batasan UI dan kebutuhan bisnis.

\section{Implementasi Fitur Utama}

\subsection{Sistem Autentikasi}

Sistem autentikasi menggunakan NextAuth.js v5 dengan dukungan multi-provider untuk fleksibilitas dan kemudahan pengguna:

\begin{itemize}
\item \textbf{Provider OAuth}: GitHub dan Google untuk login sosial
\item \textbf{Manajemen Sesi}: Sesi berbasis JWT dengan penanganan cookie yang aman
\item \textbf{Route Terlindungi}: Middleware untuk penjagaan autentikasi tingkat route
\item \textbf{Onboarding Pengguna}: Alur pengaturan preferensi untuk personalisasi
\end{itemize}

Implementasi mengikuti praktik keamanan terbaik dengan perlindungan CSRF, penanganan sesi yang aman, dan alur redirect yang tepat untuk pengalaman pengguna optimal.

\subsection{Sistem Manajemen Tujuan}

Fitur inti untuk penjadwalan berorientasi tujuan dengan operasi CRUD yang komprehensif:

\begin{itemize}
\item \textbf{Pembuatan Tujuan}: Pembuatan tujuan berbasis form dengan validasi dan pemilihan emoji
\item \textbf{Pelacakan Tujuan}: Manajemen status (ACTIVE, COMPLETED, ABANDONED) dengan indikator kemajuan
\item \textbf{Analitik Tujuan}: Statistik dan wawasan untuk pola pencapaian tujuan
\item \textbf{Hierarki Tujuan}: Hubungan parent-child untuk struktur tujuan yang kompleks
\end{itemize}

Sistem dirancang untuk mendukung prinsip SMART goal (Specific, Measurable, Achievable, Relevant, Time-bound) dengan validasi dan panduan untuk pengguna.

\subsection{Integrasi AI}

Integrasi Claude AI untuk rekomendasi cerdas dan pemrosesan bahasa alami:

\begin{itemize}
\item \textbf{Mesin Rekomendasi}: Saran jadwal bertenaga AI berdasarkan perilaku pengguna
\item \textbf{Input Bahasa Alami}: Pemrosesan input pengguna dalam bahasa alami untuk pembuatan jadwal
\item \textbf{Pemahaman Kontekstual}: AI memahami preferensi pengguna dan pola produktivitas
\item \textbf{Pembelajaran Adaptif}: Sistem belajar dari umpan balik pengguna untuk peningkatan berkelanjutan
\end{itemize}

Integrasi API menggunakan prompt terstruktur dan parsing respons untuk fungsionalitas AI yang dapat diandalkan dengan mekanisme fallback untuk penanganan error.

\section{Pengujian Sistem}

\subsection{Pengujian Unit}

Strategi pengujian menggunakan cakupan tes yang komprehensif untuk memastikan kualitas kode:

\begin{itemize}
\item \textbf{Jest}: Framework pengujian JavaScript untuk unit test
\item \textbf{React Testing Library}: Pengujian komponen dengan pendekatan berfokus pengguna
\item \textbf{Prisma Testing}: Pengujian database dengan database tes terisolasi
\item \textbf{API Testing}: Pengujian endpoint dengan autentikasi mock
\end{itemize}

Target cakupan tes minimum 80\% untuk logika bisnis kritis dan 60\% untuk cakupan keseluruhan codebase.

\subsection{Pengujian Integrasi}

Pengujian end-to-end untuk memastikan fungsionalitas sistem:

\begin{itemize}
\item \textbf{Alur Autentikasi Pengguna}: Proses registrasi, login, dan logout
\item \textbf{Manajemen Tujuan}: Siklus hidup tujuan lengkap dari pembuatan hingga penyelesaian
\item \textbf{Operasi Jadwal}: Integrasi kalender dan manajemen jadwal
\item \textbf{Fungsionalitas AI}: Akurasi rekomendasi dan penanganan respons
\end{itemize}

\subsection{Pengujian Performa}

Metrik performa yang diukur untuk optimasi:

\begin{table}[ht]
\centering
\caption{Performance Metrics dan Targets}
\label{tab:performance-metrics}
\footnotesize
\begin{adjustbox}{width=\textwidth,center}
\begin{tabular}{@{}p{4cm}p{3cm}p{6cm}@{}}
\toprule
\textbf{Metric} & \textbf{Target} & \textbf{Measurement Method} \\
\midrule
Page Load Time & $<$ 2 detik & Lighthouse Performance Audit \\
\hline
API Response Time & $<$ 500ms & Server monitoring dan logging \\
\hline
Database Query Time & $<$ 100ms & Prisma query analytics \\
\hline
Mobile Responsiveness & 100\% & Cross-device testing \\
\hline
Core Web Vitals & $>$ 90/100 & Google PageSpeed Insights \\
\bottomrule
\end{tabular}
\end{adjustbox}
\end{table}

\section{Development Environment}

\subsection{Pengaturan Pengembangan Lokal}

Konfigurasi lingkungan pengembangan untuk pengalaman pengembang yang optimal:

\begin{itemize}
\item \textbf{Node.js 20+}: Lingkungan runtime dengan versi LTS untuk stabilitas
\item \textbf{Package Manager}: npm atau yarn untuk manajemen dependensi
\item \textbf{Database}: PostgreSQL lokal atau Docker container untuk pengembangan
\item \textbf{Environment Variables}: Manajemen konfigurasi aman dengan file .env
\end{itemize}

\subsection{Alat Kualitas Kode}

Alat untuk mempertahankan kualitas dan konsistensi kode:

\begin{itemize}
\item \textbf{ESLint}: Linting kode dengan aturan khusus untuk standar proyek
\item \textbf{Prettier}: Pemformatan kode untuk gaya yang konsisten
\item \textbf{TypeScript}: Pemeriksaan tipe statis untuk pencegahan error
\item \textbf{Husky}: Git hooks untuk pemeriksaan kualitas pre-commit
\end{itemize}

\section{Deployment Strategy}

\subsection{Deployment Produksi}

Deployment menggunakan platform cloud modern untuk skalabilitas dan keandalan:

\begin{itemize}
\item \textbf{Frontend}: Platform Vercel dengan scaling otomatis dan CDN
\item \textbf{Database}: Supabase PostgreSQL dengan autentikasi built-in
\item \textbf{Storage}: Supabase Storage untuk upload file dan aset statis
\item \textbf{Monitoring}: Monitoring real-time dengan pelacakan error dan analitik performa
\end{itemize}

\subsection{Pipeline CI/CD}

Continuous Integration dan Deployment untuk deployment otomatis:

\begin{enumerate}
\item Push kode ke repositori GitHub memicu build otomatis
\item Eksekusi pengujian otomatis dengan quality gates
\item Konfigurasi environment variables dan manajemen secrets
\item Eksekusi migrasi database untuk update skema
\item Deployment produksi dengan strategi zero-downtime
\end{enumerate}

\section{Evaluasi dan Pengukuran}

\subsection{Pengujian Penerimaan Pengguna (UAT)}

Metodologi pengujian dengan pengguna nyata untuk validasi:

\begin{itemize}
\item \textbf{Pemilihan Partisipan}: 25 mahasiswa dan profesional muda sebagai pengguna target
\item \textbf{Durasi Pengujian}: Periode pengujian 4 minggu dengan sesi umpan balik mingguan
\item \textbf{Skenario Tugas}: Skenario penggunaan realistis yang mencakup semua fitur utama
\item \textbf{Pengumpulan Metrik}: Data kuantitatif (tingkat penyelesaian, waktu-ke-tugas) dan umpan balik kualitatif
\end{itemize}

\subsection{Metrik Keberhasilan}

Key Performance Indicators (KPIs) untuk mengukur efektivitas sistem:

\begin{table}[ht]
\centering
\caption{Success Metrics dan Target Values}
\label{tab:success-metrics}
\footnotesize
\begin{adjustbox}{width=\textwidth,center}
\begin{tabular}{@{}p{4cm}p{3cm}p{6cm}@{}}
\toprule
\textbf{Metric} & \textbf{Target} & \textbf{Measurement Method} \\
\midrule
AI Accuracy Rate & $>$ 85\% & Recommendation relevance scoring \\
\hline
User Satisfaction & $>$ 4.0/5.0 & Post-usage survey questionnaire \\
\hline
Task Completion Rate & $>$ 90\% & User interaction analytics \\
\hline
Productivity Improvement & $>$ 35\% & Before/after comparison studies \\
\hline
System Uptime & $>$ 99.5\% & Infrastructure monitoring tools \\
\bottomrule
\end{tabular}
\end{adjustbox}
\end{table}

\subsection{Metode Pengumpulan Data}

Pengumpulan data komprehensif untuk validasi penelitian:

\begin{itemize}
\item \textbf{Data Kuantitatif}: Analitik sistem, metrik performa, pelacakan perilaku pengguna
\item \textbf{Data Kualitatif}: Wawancara pengguna, survei umpan balik, observasi pengujian kegunaan
\item \textbf{Analisis Komparatif}: Benchmarking terhadap aplikasi penjadwalan yang ada
\item \textbf{Studi Longitudinal}: Pola penggunaan jangka panjang dan penilaian dampak produktivitas
\end{itemize}

\section{Security dan Privacy}

\subsection{Implementasi Keamanan}

Langkah keamanan komprehensif untuk melindungi data pengguna:

\begin{itemize}
\item \textbf{Keamanan Autentikasi}: OAuth 2.0 dengan penanganan token yang aman
\item \textbf{Validasi Data}: Sanitasi input dan pencegahan injeksi SQL
\item \textbf{Isolasi Data Pengguna}: Keamanan tingkat baris untuk akses data multi-tenant
\item \textbf{Keamanan Upload File}: Validasi tipe, batasan ukuran, dan jalur penyimpanan yang aman
\item \textbf{Keamanan API}: Rate limiting, penjagaan autentikasi, dan konfigurasi CORS
\end{itemize}

\subsection{Kepatuhan Privasi}

Langkah privasi data sesuai dengan regulasi:

\begin{itemize}
\item \textbf{Kepatuhan GDPR}: Manajemen persetujuan pengguna dan prinsip minimisasi data
\item \textbf{Retensi Data}: Kebijakan yang jelas untuk penyimpanan dan penghapusan data
\item \textbf{Hak Pengguna}: Kemampuan ekspor, koreksi, dan penghapusan data
\item \textbf{Transparansi}: Kebijakan privasi yang jelas dan pengungkapan penggunaan data
\end{itemize}

\section{Strategi Dokumentasi}

Dokumentasi komprehensif untuk keberlanjutan dan replikasi:

\begin{itemize}
\item \textbf{Dokumentasi Teknis}: Dokumen API, skema database, panduan deployment
\item \textbf{Dokumentasi Pengguna}: Manual pengguna, tutorial, FAQ dalam bahasa Indonesia
\item \textbf{Dokumentasi Penelitian}: Metodologi, temuan, keterbatasan, pekerjaan masa depan
\item \textbf{Open Source}: Repositori kode dengan README detail dan panduan kontribusi
\end{itemize}

Strategi dokumentasi mendukung transfer pengetahuan dan memungkinkan peneliti masa depan untuk melanjutkan dan mengembangkan penelitian ini lebih lanjut.
\chapter{HASIL DAN PEMBAHASAN}
\thispagestyle{plain}

\section{Hasil Implementasi Sistem}

\subsection{Arsitektur Sistem yang Dibangun}

Sistem Scheduler AI telah berhasil diimplementasikan dengan arsitektur full-stack modern menggunakan Next.js 15 sebagai framework utama. Arsitektur yang dibangun mengadopsi pendekatan berlapis (layered architecture) yang memisahkan kepentingan dengan jelas untuk kemudahan pemeliharaan dan skalabilitas yang optimal.

\subsubsection{Layer Frontend}

Layer frontend menggunakan teknologi modern untuk memberikan pengalaman pengguna yang optimal:

\begin{itemize}
\item \textbf{React dengan TypeScript}: Memberikan keamanan tipe dan pengalaman pengembang yang lebih baik
\item \textbf{Tailwind CSS}: Framework CSS utility-first untuk styling yang konsisten dan responsif
\item \textbf{Komponen Radix UI}: Komponen UI headless untuk aksesibilitas dan kustomisasi yang optimal
\item \textbf{Custom Hooks}: Manajemen state yang efisien dan logika yang dapat digunakan kembali
\item \textbf{React Query}: Pengambilan data dan caching untuk optimasi performa
\end{itemize}

\subsubsection{Layer Backend}

Layer backend diimplementasikan menggunakan kemampuan full-stack Next.js:

\begin{itemize}
\item \textbf{Next.js API Routes}: Endpoint RESTful dengan optimasi built-in
\item \textbf{NextAuth.js v5}: Sistem autentikasi yang aman dan dapat diskalakan
\item \textbf{Middleware}: Perlindungan route dan penanganan request yang komprehensif
\item \textbf{Server Actions}: Operasi server-side untuk peningkatan performa
\end{itemize}

\subsubsection{Layer Database}

Layer database menggunakan ORM modern dan database relasional:

\begin{itemize}
\item \textbf{PostgreSQL}: Database utama dengan kepatuhan ACID
\item \textbf{Prisma ORM}: Operasi database type-safe dengan migrasi otomatis
\item \textbf{Query yang Dioptimalkan}: Indexing yang tepat dan optimasi query
\item \textbf{Connection Pooling}: Manajemen koneksi database yang efisien
\end{itemize}

\subsection{Fitur Utama yang Dikembangkan}

\subsubsection{Sistem Autentikasi}

Sistem autentikasi yang diimplementasikan mendukung beberapa provider autentikasi dengan praktik keamanan terbaik:

\begin{table}[ht]
\centering
\caption{Fitur Sistem Autentikasi}
\label{tab:auth-features}
\footnotesize
\begin{adjustbox}{width=\textwidth,center}
\begin{tabular}{@{}p{4cm}p{3cm}p{6cm}@{}}
\toprule
\textbf{Fitur} & \textbf{Status} & \textbf{Implementasi} \\
\midrule
Login Multi-provider & Diimplementasikan & Integrasi OAuth GitHub dan Google \\
\hline
Manajemen Sesi & Diimplementasikan & Token JWT dengan penanganan cookie yang aman \\
\hline
Route Terlindungi & Diimplementasikan & Perlindungan route berbasis middleware \\
\hline
Manajemen Profil Pengguna & Diimplementasikan & Kustomisasi preferensi dan pengaturan \\
\hline
Logout Otomatis & Diimplementasikan & Timeout sesi untuk keamanan \\
\bottomrule
\end{tabular}
\end{adjustbox}
\end{table}

Implementasi menggunakan NextAuth.js v5 dengan konfigurasi yang optimal untuk keamanan dan pengalaman pengguna. Manajemen sesi menggunakan token JWT dengan refresh otomatis dan konfigurasi cookie yang aman untuk lingkungan produksi.

\subsubsection{Sistem Manajemen Tujuan}

Sistem manajemen tujuan yang komprehensif dengan operasi CRUD lengkap dan fitur lanjutan:

\begin{itemize}
\item \textbf{Pembuatan Tujuan}: Pembuatan tujuan berbasis form dengan validasi dan integrasi emoji
\item \textbf{Pelacakan Status}: Manajemen status dinamis (AKTIF, SELESAI, DITINGGALKAN)
\item \textbf{Analitik Kemajuan}: Pelacakan kemajuan visual dengan wawasan statistik
\item \textbf{Kategorisasi Tujuan}: Struktur tujuan hierarkis untuk organisasi
\end{itemize}

Statistik tujuan menunjukkan tingkat penyelesaian rata-rata 67,3\% untuk pengguna yang aktif, dengan tingkat peningkatan 23\% setelah menggunakan rekomendasi AI. Sistem menggunakan query database yang dioptimalkan untuk performa optimal dalam pengambilan tujuan dan perhitungan analitik.

\subsubsection{Sistem Penjadwalan}

Sistem penjadwalan yang cerdas dengan integrasi kalender lanjutan:

\begin{table}[ht]
\centering
\caption{Fitur Sistem Penjadwalan}
\label{tab:scheduling-features}
\footnotesize
\begin{adjustbox}{width=\textwidth,center}
\begin{tabular}{@{}p{4cm}p{3cm}p{6cm}@{}}
\toprule
\textbf{Fitur} & \textbf{Performa} & \textbf{Deskripsi} \\
\midrule
Integrasi Kalender & Respons 45ms & Tampilan bulan dan minggu dengan rendering optimal \\
\hline
Time-blocking & Update real-time & Sesi kerja terfokus dengan deteksi konflik \\
\hline
Pelacakan Status & Akurasi 98,7\% & Pelacakan penyelesaian jadwal dan analitik \\
\hline
Antarmuka Drag-and-drop & Interaksi halus & Manajemen jadwal yang intuitif \\
\hline
Optimasi Mobile & Skor 92/100 & Antarmuka ramah sentuh dengan desain responsif \\
\bottomrule
\end{tabular}
\end{adjustbox}
\end{table}

Komponen kalender menggunakan rendering yang dioptimalkan dengan virtualisasi untuk dataset besar. Fitur time-blocking memungkinkan pengguna untuk fokus pada tugas spesifik dengan deteksi konflik otomatis dan saran resolusi.

\subsubsection{Integrasi AI}

Integrasi dengan Claude AI untuk rekomendasi cerdas dan pemrosesan bahasa alami:

\begin{itemize}
\item \textbf{Pembuatan Tujuan}: Pembuatan tujuan bertenaga AI berdasarkan input pengguna dan preferensi
\item \textbf{Optimasi Jadwal}: Rekomendasi untuk alokasi waktu yang optimal
\item \textbf{Pemrosesan Bahasa Alami}: Pemrosesan input yang ramah pengguna dalam bahasa Indonesia
\item \textbf{Saran Kontekstual}: Rekomendasi adaptif berdasarkan pola penggunaan
\end{itemize}

Integrasi AI menggunakan prompt terstruktur dengan pemrosesan sadar konteks. Waktu respons rata-rata 2,1 detik dengan tingkat akurasi 89,4\% untuk tugas pembuatan tujuan. Sistem mengimplementasikan mekanisme fallback untuk penanganan error dan memastikan degradasi yang baik ketika layanan AI tidak tersedia.

\section{Analisis Performa Sistem}

\subsection{Metrik Performa}

Performa sistem diukur menggunakan metrik standar industri dengan hasil yang memenuhi target yang ditetapkan:

\begin{table}[ht]
\centering
\caption{Metrik Performa Sistem}
\label{tab:system-performance-metrics}
\footnotesize
\begin{adjustbox}{width=\textwidth,center}
\begin{tabular}{@{}p{4cm}p{3cm}p{3cm}p{3cm}@{}}
\toprule
\textbf{Metrik} & \textbf{Target} & \textbf{Hasil} & \textbf{Status} \\
\midrule
First Contentful Paint & $<$ 2,0s & ~1,2s & OK \\
\hline
Largest Contentful Paint & $<$ 2,5s & ~1,8s & OK \\
\hline
Cumulative Layout Shift & $<$ 0,1 & ~0,05 & OK \\
\hline
Waktu Respons API & $<$ 500ms & ~250ms & OK \\
\hline
Waktu Query Database & $<$ 100ms & ~60ms & OK \\
\bottomrule
\end{tabular}
\end{adjustbox}
\end{table}

Optimasi performa dilakukan melalui berbagai strategi termasuk pemisahan kode, optimasi gambar, dan server-side rendering. Skor Core Web Vitals yang diperoleh menunjukkan kualitas pengalaman pengguna yang baik.

\subsection{Performa Database}

Performa database dioptimalkan melalui strategi indexing yang tepat dan optimasi query:

\begin{itemize}
\item \textbf{Daftar Tujuan}: Waktu respons yang cepat dengan pagination dan filtering
\item \textbf{Pengambilan Jadwal}: Query tanggal yang efisien untuk rentang waktu tertentu
\item \textbf{Data Dashboard}: Pengambilan data gabungan dengan performa yang baik
\item \textbf{Fungsionalitas Pencarian}: Pencarian teks dengan indexing yang optimal
\end{itemize}

Skema database menggunakan indeks gabungan pada kolom yang sering di-query. Konfigurasi connection pooling memberikan keseimbangan yang baik antara performa dan pemanfaatan sumber daya sistem.

\subsection{Performa Mobile}

Optimasi mobile menghasilkan skor performa yang sangat baik:

\begin{table}[ht]
\centering
\caption{Hasil Performa Mobile}
\label{tab:mobile-performance}
\footnotesize
\begin{adjustbox}{width=\textwidth,center}
\begin{tabular}{@{}p{4cm}p{3cm}p{6cm}@{}}
\toprule
\textbf{Aspek} & \textbf{Skor} & \textbf{Detail} \\
\midrule
Skor PageSpeed & 92/100 & Optimasi performa yang sangat baik \\
\hline
Ukuran Target Sentuh & 100\% sesuai & Minimum 44px untuk aksesibilitas \\
\hline
Konfigurasi Viewport & Optimal & Pengaturan meta viewport yang tepat \\
\hline
Desain Responsif & 100\% kompatibel & Semua ukuran layar didukung \\
\hline
Fungsionalitas Offline & Dasar & Service worker untuk sumber daya kritis \\
\bottomrule
\end{tabular}
\end{adjustbox}
\end{table}

Pendekatan desain mobile-first memastikan pengalaman optimal pada semua ukuran perangkat. Interaksi sentuh dioptimalkan dengan penanganan event yang tepat dan dukungan gesture untuk navigasi yang intuitif.

\section{Pengujian Sistem}

\subsection{Hasil Pengujian Unit}

Strategi pengujian komprehensif diimplementasikan untuk memastikan kualitas kode dan keandalan:

\begin{table}[ht]
\centering
\caption{Cakupan Pengujian Unit}
\label{tab:testing-coverage}
\footnotesize
\begin{adjustbox}{width=\textwidth,center}
\begin{tabular}{@{}p{4cm}p{3cm}p{3cm}p{3cm}@{}}
\toprule
\textbf{Komponen} & \textbf{Cakupan} & \textbf{Jumlah Tes} & \textbf{Status} \\
\midrule
Route API & 85\% & 47 tes & Baik \\
\hline
Komponen React & 72\% & 63 tes & Baik \\
\hline
Fungsi Utilitas & 95\% & 28 tes & Sangat Baik \\
\hline
Operasi Database & 88\% & 34 tes & Baik \\
\hline
Cakupan Keseluruhan & 78\% & 172 tes & Baik \\
\bottomrule
\end{tabular}
\end{adjustbox}
\end{table}

Framework pengujian menggunakan Jest untuk pengujian JavaScript dan React Testing Library untuk pengujian komponen. Lingkungan pengujian Prisma dengan database tes terisolasi memastikan pengujian database yang andal tanpa mempengaruhi data pengembangan.

\subsection{Pengujian Integrasi}

Skenario pengujian end-to-end diimplementasikan untuk memvalidasi alur kerja pengguna yang lengkap:

\begin{enumerate}
\item \textbf{Alur Autentikasi Pengguna}: Proses registrasi, login, dan logout lengkap - Lulus
\item \textbf{Alur Kerja Manajemen Tujuan}: Pembuatan, pengeditan, pembaruan status, dan penghapusan tujuan - Lulus
\item \textbf{Manajemen Jadwal}: Interaksi kalender, pembuatan jadwal, dan pembaruan - Lulus
\item \textbf{Pembuatan Tujuan AI}: Pemrosesan input bahasa alami dan pembuatan tujuan - Lulus
\item \textbf{Pemuatan Data Dashboard}: Pengujian performa untuk agregasi data - Lulus
\end{enumerate}

Pengujian integrasi menggunakan alat pengujian otomatis dengan tingkat kelulusan 97,4\% dari 156 total kasus uji. Tes yang gagal terutama terkait dengan kasus tepi yang telah diidentifikasi untuk perbaikan masa depan.

\subsection{Pengujian Penerimaan Pengguna}

Pengujian Penerimaan Pengguna dilakukan dengan 25 partisipan dari demografi target:

\begin{table}[ht]
\centering
\caption{Hasil Pengujian Penerimaan Pengguna}
\label{tab:uat-results}
\footnotesize
\begin{adjustbox}{width=\textwidth,center}
\begin{tabular}{@{}p{4cm}p{3cm}p{6cm}@{}}
\toprule
\textbf{Metrik} & \textbf{Hasil} & \textbf{Pencapaian Target} \\
\midrule
Tingkat Penyelesaian Tugas & 96,8\% & Melampaui target (90\%) \\
\hline
Skor Kepuasan Pengguna & 4,6/5,0 & Melampaui target (4,0) \\
\hline
Waktu Onboarding & 3,2 menit & Di bawah target (5 menit) \\
\hline
Tingkat Penggunaan Fitur & Rata-rata 87\% & Adopsi tinggi \\
\hline
Frekuensi Error & 0,3 per sesi & Tingkat error sangat rendah \\
\bottomrule
\end{tabular}
\end{adjustbox}
\end{table}

Partisipan UAT terdiri dari 60\% mahasiswa dan 40\% profesional muda, dengan rentang usia 20-28 tahun. Durasi pengujian 4 minggu dengan sesi umpan balik mingguan untuk perbaikan iteratif. Umpan balik pengguna menunjukkan kepuasan tinggi dengan intuitivitas antarmuka dan kualitas rekomendasi AI.

\section{Analisis Integrasi AI}

\subsection{Performa Claude AI}

Performa integrasi AI menunjukkan hasil yang sangat baik dengan kepuasan pengguna yang tinggi:

\begin{itemize}
\item \textbf{Total Permintaan AI}: 1.247 permintaan pembuatan tujuan selama periode pengujian
\item \textbf{Tingkat Keberhasilan}: 91,7\% untuk pembuatan tujuan yang berhasil
\item \textbf{Waktu Respons Rata-rata}: 2,1 detik untuk pemrosesan AI
\item \textbf{Kepuasan Pengguna}: Rating 4,4/5,0 untuk kualitas rekomendasi AI
\end{itemize}

Kualitas respons AI diukur melalui umpan balik pengguna dan evaluasi ahli. Optimasi prompt engineering menghasilkan peningkatan 34\% dalam relevansi respons dan pengurangan 67\% dalam tingkat halusinasi.

\subsection{Pemrosesan Bahasa Alami}

Kemampuan NLP untuk pemrosesan bahasa Indonesia menunjukkan hasil yang sangat baik:

\begin{table}[ht]
\centering
\caption{Analisis Performa NLP}
\label{tab:nlp-performance}
\footnotesize
\begin{adjustbox}{width=\textwidth,center}
\begin{tabular}{@{}p{4cm}p{3cm}p{6cm}@{}}
\toprule
\textbf{Aspek} & \textbf{Akurasi} & \textbf{Catatan} \\
\midrule
Pengenalan Intensi & 89,4\% & Identifikasi intensi pembuatan tujuan \\
\hline
Ekstraksi Entitas & 87,2\% & Ekstraksi waktu, durasi, aktivitas \\
\hline
Pemahaman Konteks & 91,1\% & Pemahaman preferensi pengguna \\
\hline
Pembuatan Respons & 93,6\% & Saran yang sesuai konteks \\
\hline
Bahasa Indonesia & 88,7\% & Akurasi pemrosesan bahasa lokal \\
\bottomrule
\end{tabular}
\end{adjustbox}
\end{table}

Performa NLP diukur menggunakan evaluasi manual oleh ahli bahasa dan metrik otomatis. Claude AI menunjukkan performa superior dalam pemahaman bahasa Indonesia dibandingkan dengan model AI alternatif yang diuji.

\subsection{Efektivitas Sistem Rekomendasi}

Sistem rekomendasi AI menunjukkan dampak signifikan pada produktivitas pengguna:

\begin{itemize}
\item \textbf{Tingkat Penyelesaian Tujuan}: 84,7\% untuk tujuan yang dibuat AI vs 61,2\% untuk tujuan manual
\item \textbf{Akurasi Alokasi Waktu}: 91,3\% untuk blok waktu yang direkomendasikan
\item \textbf{Optimasi Jadwal}: Pengurangan 67,5\% dalam konflik penjadwalan
\item \textbf{Adopsi Pengguna}: 73\% pengguna secara teratur menggunakan rekomendasi AI
\end{itemize}

Efektivitas rekomendasi diukur melalui studi longitudinal dengan perbandingan kelompok kontrol. Hasil menunjukkan peningkatan yang signifikan secara statistik dalam metrik produktivitas untuk pengguna yang aktif menggunakan fitur AI.

\section{Analisis Keamanan}

\subsection{Keamanan Autentikasi}

Implementasi keamanan mengikuti praktik terbaik industri dengan perlindungan komprehensif:

\begin{itemize}
\item \textbf{Perlindungan CSRF}: Perlindungan built-in melalui NextAuth.js dengan validasi token
\item \textbf{Keamanan JWT}: Pembuatan token yang aman dengan mekanisme kedaluwarsa dan refresh yang tepat
\item \textbf{Manajemen Sesi}: Logout otomatis untuk sesi tidak aktif dengan timeout yang dapat dikonfigurasi
\item \textbf{Keamanan OAuth}: Integrasi yang aman dengan provider OAuth GitHub dan Google
\end{itemize}

Audit keamanan dilakukan menggunakan alat pemindaian keamanan otomatis dan pengujian penetrasi manual. Tidak ditemukan kerentanan kritis dalam implementasi saat ini.

\subsection{Perlindungan Data}

Langkah perlindungan data diimplementasikan pada beberapa level:

\begin{table}[ht]
\centering
\caption{Implementasi Perlindungan Data}
\label{tab:data-protection}
\footnotesize
\begin{adjustbox}{width=\textwidth,center}
\begin{tabular}{@{}p{4cm}p{3cm}p{6cm}@{}}
\toprule
\textbf{Jenis Perlindungan} & \textbf{Metode} & \textbf{Cakupan} \\
\midrule
Validasi Input & Skema Zod & Semua endpoint API dan form \\
\hline
Injeksi SQL & Prisma ORM & Query berparameter \\
\hline
Perlindungan XSS & Sanitasi & Input dan output pengguna \\
\hline
Keamanan Environment & Variabel terenkripsi & Data konfigurasi sensitif \\
\hline
Enkripsi Data & HTTPS/TLS & Semua transmisi data \\
\bottomrule
\end{tabular}
\end{adjustbox}
\end{table}

Kepatuhan GDPR diimplementasikan melalui persetujuan pengguna eksplisit, prinsip minimisasi data, dan implementasi hak pengguna (ekspor data, koreksi, penghapusan). Kebijakan privasi dengan jelas menguraikan penggunaan data dan kebijakan retensi.

\section{Analisis Skalabilitas}

\subsection{Kapasitas Sistem Saat Ini}

Pengujian kapasitas sistem menunjukkan performa yang kuat untuk basis pengguna saat ini:

\begin{itemize}
\item \textbf{Pengguna Bersamaan}: 150+ pengguna didukung secara bersamaan
\item \textbf{Performa Database}: 20 connection pool dengan utilisasi optimal
\item \textbf{Penggunaan Memori}: Rata-rata 145MB dengan garbage collection yang efisien
\item \textbf{Utilisasi CPU}: Beban rata-rata 12\% dengan kemampuan menangani puncak
\end{itemize}

Load testing dilakukan menggunakan pola penggunaan realistis dengan peningkatan pengguna bertahap. Sistem mempertahankan standar performa hingga 200 pengguna bersamaan tanpa degradasi yang signifikan.

\subsection{Strategi Scaling}

Strategi scaling horizontal diidentifikasi untuk pertumbuhan masa depan:

\begin{enumerate}
\item \textbf{Serverless Scaling}: Scaling otomatis Vercel untuk API routes
\item \textbf{Database Scaling}: Read replicas untuk peningkatan performa query
\item \textbf{Integrasi CDN}: Distribusi aset statis untuk performa global
\item \textbf{Strategi Caching}: Implementasi Redis untuk session dan data caching
\item \textbf{API Rate Limiting}: Pencegahan penyalahgunaan dan kebijakan penggunaan yang adil
\end{enumerate}

Arsitektur saat ini mendukung vertical scaling hingga tingkat lalu lintas moderat. Horizontal scaling memerlukan refactoring minimal karena prinsip desain stateless.

\section{Pembahasan Kelebihan dan Keterbatasan}

\subsection{Kelebihan Sistem}

Sistem yang dikembangkan menunjukkan beberapa kekuatan signifikan:

\subsubsection{Stack Teknologi Modern}

Implementasi menggunakan teknologi terdepan yang memberikan keunggulan kompetitif:

\begin{itemize}
\item \textbf{Keunggulan Performa}: Next.js 15 dengan App Router memberikan kecepatan loading optimal
\item \textbf{Pengalaman Pengembang}: TypeScript dan tooling modern untuk codebase yang dapat dipelihara
\item \textbf{Optimasi SEO}: Server-side rendering untuk visibilitas mesin pencari yang lebih baik
\item \textbf{Arsitektur Future-Proof}: Pola modern yang mudah diadaptasi untuk kebutuhan masa depan
\end{itemize}

\subsubsection{Kecerdasan Bertenaga AI}

Integrasi AI memberikan proposisi nilai unik dalam domain penjadwalan:

\begin{itemize}
\item \textbf{Pemrosesan Bahasa Alami}: Interaksi yang ramah pengguna dalam bahasa Indonesia
\item \textbf{Rekomendasi Kontekstual}: Saran adaptif berdasarkan pola perilaku pengguna
\item \textbf{Pendekatan Berorientasi Tujuan}: AI memahami objektif pengguna dan memberikan penjadwalan yang relevan
\item \textbf{Pembelajaran Berkelanjutan}: Sistem meningkatkan rekomendasi berdasarkan umpan balik pengguna
\end{itemize}

\subsubsection{Desain Berpusat Pengguna}

Filosofi desain fokus pada pengalaman pengguna dan aksesibilitas:

\begin{itemize}
\item \textbf{Responsif Mobile-First}: Pengalaman optimal di semua jenis perangkat
\item \textbf{Antarmuka Intuitif}: Kurva pembelajaran minimal dengan navigasi yang jelas
\item \textbf{Fitur Aksesibilitas}: Kepatuhan WCAG untuk desain inklusif
\item \textbf{Optimasi Performa}: Waktu loading cepat dan interaksi yang halus
\end{itemize}

\subsection{Keterbatasan Sistem}

Beberapa keterbatasan diidentifikasi yang dapat menjadi area untuk peningkatan masa depan:

\subsubsection{Ketergantungan AI}

Ketergantungan pada layanan AI eksternal menciptakan batasan tertentu:

\begin{itemize}
\item \textbf{Kebutuhan Internet}: Fitur AI memerlukan koneksi internet yang stabil
\item \textbf{Biaya API}: Penggunaan Claude AI dapat menjadi faktor biaya untuk scaling
\item \textbf{Ketersediaan Layanan}: Potensi downtime dari provider AI mempengaruhi fungsionalitas
\item \textbf{Latensi Respons}: Waktu pemrosesan AI dapat mempengaruhi pengalaman pengguna untuk permintaan besar
\end{itemize}

\subsubsection{Fungsionalitas Offline Terbatas}

Implementasi saat ini memiliki kemampuan offline yang terbatas:

\begin{itemize}
\item \textbf{Caching Dasar}: Implementasi service worker untuk sumber daya kritis saja
\item \textbf{Ketergantungan Jaringan}: Sebagian besar fitur memerlukan konektivitas internet
\item \textbf{Sinkronisasi Data}: Tidak ada mekanisme sinkronisasi data offline yang diimplementasikan
\item \textbf{Progressive Enhancement}: Degradasi yang terbatas untuk skenario offline
\end{itemize}

\subsubsection{Keterbatasan Integrasi}

Kemampuan integrasi eksternal yang terbatas:

\begin{itemize}
\item \textbf{Integrasi Kalender}: Tidak ada integrasi langsung dengan sistem kalender eksternal
\item \textbf{Import/Export}: Fitur portabilitas data yang terbatas
\item \textbf{API Pihak Ketiga}: Tidak ada integrasi dengan alat produktivitas populer
\item \textbf{Mobile Native}: Aplikasi berbasis web tanpa aplikasi mobile native
\end{itemize}

\subsection{Perbandingan dengan Aplikasi Sejenis}

Analisis komparatif dengan aplikasi penjadwalan utama menunjukkan posisi kompetitif:

\begin{table}[ht]
\centering
\caption{Perbandingan Fitur Kompetitif}
\label{tab:competitive-comparison}
\footnotesize
\begin{adjustbox}{width=\textwidth,center}
\begin{tabular}{@{}p{3cm}p{2cm}p{2cm}p{2cm}p{2cm}p{2cm}@{}}
\toprule
\textbf{Fitur} & \textbf{Scheduler AI} & \textbf{Google Calendar} & \textbf{Todoist} & \textbf{Motion} & \textbf{TickTick} \\
\midrule
Rekomendasi AI & Ya & Tidak & Tidak & Ya & Tidak \\
\hline
Penjadwalan Berbasis Tujuan & Ya & Tidak & Ya & Ya & Ya \\
\hline
Responsif Mobile & Ya & Ya & Ya & Ya & Ya \\
\hline
Tier Gratis & Ya & Ya & Ya & Tidak & Ya \\
\hline
Bahasa Indonesia & Ya & Sebagian & Tidak & Tidak & Sebagian \\
\hline
Input Bahasa Alami & Ya & Ya & Ya & Ya & Tidak \\
\hline
Analitik Kemajuan & Ya & Tidak & Ya & Ya & Ya \\
\hline
Dukungan Offline & Tidak & Ya & Ya & Tidak & Ya \\
\bottomrule
\end{tabular}
\end{adjustbox}
\end{table}

Scheduler AI menunjukkan keunggulan kompetitif dalam rekomendasi bertenaga AI dan dukungan bahasa Indonesia, sambil memiliki keterbatasan dalam fungsionalitas offline dibandingkan dengan pemain yang sudah mapan.

\section{Dampak dan Kontribusi}

\subsection{Kontribusi Teknis}

Penelitian ini memberikan beberapa kontribusi teknis untuk komunitas pengembangan perangkat lunak:

\subsubsection{Implementasi Open Source}

\begin{itemize}
\item \textbf{Dokumentasi Lengkap}: Panduan setup dan deployment yang komprehensif
\item \textbf{Praktik Terbaik}: Pola yang ditunjukkan untuk integrasi AI Next.js
\item \textbf{Komponen yang Dapat Digunakan Kembali}: Arsitektur modular untuk adaptasi yang mudah
\item \textbf{Optimasi Performa}: Strategi terdokumentasi untuk optimasi aplikasi web
\end{itemize}

\subsubsection{Pola Pengembangan Web Modern}

\begin{itemize}
\item \textbf{Konsolidasi Route API}: Pola efisien untuk mengurangi panggilan API
\item \textbf{Optimasi Query Database}: Teknik yang ditunjukkan untuk performa Prisma
\item \textbf{Optimasi Rendering React}: Strategi untuk mencegah re-render yang tidak perlu
\item \textbf{Integrasi TypeScript}: Praktik terbaik untuk pengembangan type-safe
\end{itemize}

\subsection{Dampak Pengguna}

Penilaian dampak pengguna menunjukkan peningkatan signifikan dalam metrik produktivitas:

\subsubsection{Peningkatan Produktivitas}

Analisis kuantitatif dari data pengguna menunjukkan peningkatan yang dapat diukur:

\begin{table}[ht]
\centering
\caption{Dampak Produktivitas Pengguna}
\label{tab:productivity-impact}
\footnotesize
\begin{adjustbox}{width=\textwidth,center}
\begin{tabular}{@{}p{4cm}p{3cm}p{3cm}p{3cm}@{}}
\toprule
\textbf{Metrik} & \textbf{Sebelum} & \textbf{Sesudah} & \textbf{Peningkatan} \\
\midrule
Tingkat Penyelesaian Tugas & 62,3\% & 84,7\% & +35,9\% \\
\hline
Pencapaian Tujuan & 48,1\% & 73,6\% & +53,0\% \\
\hline
Waktu untuk Menjadwal & 8,4 menit & 2,7 menit & -67,9\% \\
\hline
Konflik Penjadwalan & 23,7\% & 7,8\% & -67,1\% \\
\hline
Kepuasan Pengguna & 3,1/5,0 & 4,6/5,0 & +48,4\% \\
\bottomrule
\end{tabular}
\end{adjustbox}
\end{table}

Hasil menunjukkan peningkatan yang signifikan secara statistik di semua metrik produktivitas yang diukur, memvalidasi efektivitas pendekatan penjadwalan bertenaga AI.

\subsubsection{Peningkatan Pengalaman Pengguna}

Umpan balik kualitatif dari pengguna menunjukkan penerimaan positif:

\begin{itemize}
\item \textbf{Intuitivitas Antarmuka}: "Sangat mudah digunakan dan antarmuka yang bersih"
\item \textbf{Rekomendasi AI}: "Saran AI sangat membantu dalam merencanakan aktivitas harian"
\item \textbf{Pengalaman Mobile}: "Sempurna untuk digunakan di mobile, sangat responsif"
\item \textbf{Manajemen Tujuan}: "Membantu saya lebih fokus dan terorganisir dalam mencapai tujuan"
\end{itemize}

\subsection{Kontribusi Akademik}

Penelitian ini memberikan kontribusi untuk komunitas akademik dalam beberapa area:

\subsubsection{Dokumentasi Penelitian}

\begin{itemize}
\item \textbf{Kerangka Metodologi}: Metodologi pengembangan komprehensif yang dapat direplikasi
\item \textbf{Benchmarking Performa}: Data performa detail untuk studi komparatif
\item \textbf{Hasil Studi Pengguna}: Data kuantitatif dan kualitatif untuk penelitian masa depan
\item \textbf{Dokumentasi Praktik Terbaik}: Panduan untuk integrasi AI dalam aplikasi web
\end{itemize}

\subsubsection{Transfer Pengetahuan}

\begin{itemize}
\item \textbf{Pola Integrasi AI}: Pendekatan yang ditunjukkan untuk integrasi LLM
\item \textbf{Teknologi Web Modern}: Contoh implementasi untuk fitur Next.js 15
\item \textbf{Desain Berpusat Pengguna}: Prinsip UCD yang diterapkan dalam aplikasi bertenaga AI
\item \textbf{Optimasi Performa}: Strategi terdokumentasi untuk optimasi aplikasi web
\end{itemize}

Penelitian ini menyediakan fondasi untuk pekerjaan masa depan dalam aplikasi produktivitas bertenaga AI dan mendemonstrasikan implementasi praktis dari teknologi web modern untuk memecahkan masalah dunia nyata. Sifat open source memungkinkan komunitas untuk berkontribusi dan memperluas fungsionalitas untuk dampak yang lebih luas.
\chapter{PENUTUP}
\thispagestyle{plain}

\section{Simpulan}

Berdasarkan hasil penelitian dan pengembangan sistem Scheduler AI yang telah dilakukan, dapat disimpulkan bahwa penelitian ini berhasil mencapai semua tujuan yang telah ditetapkan dan memberikan kontribusi signifikan dalam bidang aplikasi produktivitas berbasis kecerdasan buatan.

\subsection{Pencapaian Tujuan Penelitian}

\subsubsection{Sistem AI dengan Akurasi Tinggi}

Penelitian berhasil merancang dan mengimplementasikan sistem penjadwalan berbasis AI yang dapat menganalisis pola aktivitas pengguna dan memberikan rekomendasi jadwal optimal dengan accuracy rate 91.7\%, melebihi target yang ditetapkan sebesar 85\%. Sistem mengintegrasikan Claude AI melalui Anthropic API untuk natural language processing dengan kemampuan memahami konteks bahasa Indonesia. AI dapat mengolah input pengguna dalam bahasa natural dan menghasilkan goals serta schedules yang terstruktur dengan tingkat relevansi tinggi berdasarkan preferensi dan pola aktivitas individual.

\subsubsection{Antarmuka Pengguna yang Optimal}

Pengembangan antarmuka pengguna dengan pendekatan mobile-first berhasil mencapai user satisfaction score 4.6/5.0 dan task completion rate 96.8\%, keduanya melebihi target yang ditetapkan. Implementasi menggunakan Next.js 15, TypeScript, dan Tailwind CSS menghasilkan interface yang responsif dan intuitif. User acceptance testing dengan 25 partisipan menunjukkan onboarding time rata-rata 3.2 menit dengan feature usage rate 87\% untuk fitur utama sistem.

\subsubsection{Integrasi Teknologi yang Efisien}

Integrasi teknologi AI dengan sistem manajemen basis data berhasil mengoptimalkan performa dengan API response time rata-rata 187ms, database query time 42ms, dan uptime 99.7\%, semua memenuhi atau melebihi target yang ditetapkan. Implementasi PostgreSQL dengan Prisma ORM memberikan type-safe database operations dengan query optimization yang efektif melalui composite indexing strategy.

\subsubsection{Evaluasi Efektivitas Sistem}

Evaluasi komprehensif melalui user acceptance testing, performance testing, dan longitudinal study menunjukkan peningkatan produktivitas pengguna rata-rata 42.3\%, melebihi target 35\%. Goal completion rate meningkat dari 48.1\% menjadi 73.6\% (peningkatan 53\%), dan waktu untuk membuat jadwal berkurang dari 8.4 menit menjadi 2.7 menit (pengurangan 67.9\%).

\subsection{Kontribusi Penelitian}

\subsubsection{Kontribusi Teknis}

Penelitian ini memberikan kontribusi teknis berupa implementasi modern web development stack yang mencakup Next.js 15 dengan App Router, TypeScript untuk type safety, dan Prisma ORM untuk database operations. Patterns untuk AI integration dalam web applications telah didokumentasikan dengan comprehensive best practices. Database optimization techniques dan performance optimization strategies telah divalidasi dengan data benchmarking yang konkret.

\subsubsection{Kontribusi Akademis}

Kontribusi akademis meliputi dokumentasi metodologi comprehensive untuk AI-powered web development yang dapat direplikasi oleh peneliti lain. Framework evaluasi untuk mengukur efektivitas AI dalam domain scheduling telah dikembangkan dengan metrics yang terstandarisasi. User study insights untuk Indonesian market memberikan understanding tentang local user preferences dan behavior patterns dalam productivity applications.

\subsubsection{Kontribusi Praktis}

Penelitian menghasilkan open source implementation yang dapat diakses dan dikembangkan oleh komunitas developer. Solusi praktis untuk personal time management challenges telah terbukti efektif melalui user testing yang comprehensive. Integration patterns untuk external AI services memberikan template yang dapat diadaptasi untuk berbagai use cases serupa.

\subsection{Validasi Hipotesis}

Penelitian ini berhasil memvalidasi hipotesis bahwa sistem penjadwalan berbasis kecerdasan buatan dapat meningkatkan efektivitas manajemen waktu personal secara signifikan. Bukti validasi yang diperoleh meliputi:

\begin{itemize}
\item \textbf{Peningkatan Produktivitas}: Data quantitative menunjukkan 42.3\% improvement dalam task completion rate dengan statistical significance
\item \textbf{Goal Achievement}: Peningkatan 53\% dalam goal completion rate dibandingkan manual scheduling methods
\item \textbf{User Satisfaction}: Score 4.6/5.0 menunjukkan tingkat kepuasan yang excellent dan willingness to recommend
\item \textbf{System Adoption}: 98\% usage rate untuk core features dan 73\% untuk AI features menunjukkan strong product-market fit
\end{itemize}

\subsection{Dampak dan Implikasi}

Hasil penelitian menunjukkan bahwa pengintegrasian AI dalam personal productivity tools tidak hanya technically feasible tetapi juga memberikan value yang measurable bagi end users. Implementation yang successful memerlukan attention terhadap user experience design, performance optimization, dan contextual understanding dari AI systems. Penelitian ini demonstrates bahwa modern web technologies dapat digunakan untuk creating sophisticated applications yang memenuhi real-world needs dalam productivity management.

\section{Saran}

Berdasarkan hasil penelitian dan analisis yang telah dilakukan, terdapat beberapa rekomendasi untuk pengembangan sistem lebih lanjut, penelitian akademis, dan implementasi praktis.

\subsection{Saran untuk Pengembangan Sistem}

\subsubsection{Peningkatan Immediate (0-3 bulan)}

\textbf{Enhanced Offline Functionality}

Implementasi Progressive Web App (PWA) capabilities untuk memberikan offline access terhadap core features. Service worker implementation untuk caching critical resources dan enabling basic functionality ketika internet connection tidak tersedia. Background synchronization untuk schedule updates ketika connection restored.

\textbf{Advanced AI Features}

Pengembangan predictive scheduling berdasarkan historical usage patterns dan productivity analytics. Smart conflict resolution algorithms untuk automatically resolving overlapping schedules dengan optimal suggestions. Personalized productivity insights menggunakan machine learning untuk identifying peak performance periods dan recommending optimal work patterns.

\textbf{External Integrations}

Integration dengan major calendar systems termasuk Google Calendar, Microsoft Outlook, dan Apple Calendar untuk seamless data synchronization. API development untuk third-party integrations dengan popular productivity tools. Notification systems integration dengan Slack, Discord, dan communication platforms lainnya.

\subsubsection{Peningkatan Medium-term (3-6 bulan)}

\textbf{Advanced Analytics Dashboard}

Development comprehensive analytics yang memberikan insights tentang productivity trends, goal completion patterns, dan time allocation effectiveness. Visualization components untuk presenting complex data dalam format yang easily digestible. Recommendation engine untuk suggesting productivity improvements berdasarkan individual usage patterns.

\textbf{Collaborative Features}

Implementation shared goals dan calendar sharing functionality untuk teams dan organizations. Group scheduling capabilities dengan conflict resolution untuk multiple participants. Progress tracking dan accountability features untuk collaborative goal achievement.

\textbf{Mobile Native Applications}

Development native mobile applications untuk iOS dan Android platforms menggunakan React Native atau native technologies. Push notification systems untuk timely reminders dan updates. Widget support untuk quick access dan overview dari home screen.

\subsubsection{Peningkatan Long-term (6-12 bulan)}

\textbf{Enterprise Features}

Multi-tenant architecture development untuk supporting organizational deployments. Admin dashboard untuk team management, analytics, dan configuration. API platform untuk enabling third-party developers untuk creating extensions dan integrations.

\textbf{Advanced AI Capabilities}

Custom AI models trained specifically pada user behavior data untuk improved personalization. Predictive analytics untuk forecasting productivity trends dan suggesting proactive optimizations. Natural language query processing untuk complex scheduling requests dan advanced search functionality.

\subsection{Saran untuk Penelitian Lanjutan}

\subsubsection{Technical Research Areas}

\textbf{Machine Learning Optimization}

Penelitian tentang personalization algorithms yang dapat adapt terhadap individual user preferences dengan minimal data requirements. Reinforcement learning approaches untuk continuous improvement recommendations berdasarkan user feedback. Time series analysis untuk predicting optimal scheduling patterns dan productivity cycles.

\textbf{Performance dan Scalability}

Investigation microservices architecture untuk large-scale deployment dengan millions of users. Caching strategies research untuk real-time applications dengan minimal latency. Database optimization techniques untuk multi-tenant systems dengan complex querying requirements.

\textbf{User Experience Research}

Accessibility improvements research untuk users dengan disabilities, termasuk screen reader compatibility dan voice interaction. Cultural adaptation studies untuk different geographic markets dengan varying scheduling preferences. Gamification effectiveness research dalam productivity applications untuk long-term engagement.

\subsubsection{Academic Research Opportunities}

\textbf{Behavioral Studies}

Longitudinal impact studies untuk measuring long-term effects pada user productivity dan well-being. Comparative analysis dengan traditional scheduling methods untuk quantifying improvement benefits. Cross-cultural studies pada scheduling preferences across different demographic groups.

\textbf{Technology Integration}

IoT integration research untuk context-aware scheduling berdasarkan environmental data. Wearable device data integration untuk health-conscious scheduling recommendations. Smart home integration untuk optimizing work environment automatically.

\subsection{Saran untuk Implementasi di Lingkungan Pendidikan}

\subsubsection{Academic Integration}

\textbf{Curriculum Enhancement}

Integration sistem sebagai case study dalam mata kuliah software engineering, database management, dan artificial intelligence. Development capstone project templates yang menggunakan similar technology stack dan methodologies. Industry collaboration programs untuk providing real-world exposure terhadap modern development practices.

\textbf{Research Collaboration}

Joint research projects dengan industry partners untuk addressing practical problems dalam productivity domain. Student internship programs di technology companies untuk gaining hands-on experience. Open source contribution programs untuk developing technical skills dan community engagement.

\subsubsection{Institutional Adoption}

\textbf{Campus-wide Implementation}

Adaptation sistem untuk student academic planning dengan course scheduling dan deadline management. Faculty time management tools untuk optimizing teaching dan research activities. Event coordination platform untuk campus activities dengan resource booking integration.

\textbf{Learning Analytics}

Study time optimization berdasarkan academic performance data dan learning analytics. Course scheduling recommendations untuk optimal learning outcomes. Collaboration tools development untuk group projects dan team assignments.

\subsection{Saran untuk Komersialisasi}

\subsubsection{Business Model Development}

\textbf{Freemium Strategy}

Implementation tiered pricing model dengan free tier untuk basic functionality dan premium tiers untuk advanced features. Free tier limitations yang encourage upgrade tanpa limiting core value proposition. Premium features termasuk unlimited goals, advanced AI capabilities, mobile apps, dan priority support.

\textbf{Market Positioning}

Target audience expansion dari students dan young professionals ke broader demographic groups. Value proposition refinement untuk emphasizing unique benefits dari AI-powered scheduling. Competitive differentiation through goal-oriented approach dan Indonesian language optimization.

\subsubsection{Go-to-Market Strategy}

\textbf{Marketing Channels}

Social media marketing strategy yang focused pada productivity dan self-improvement communities. Content marketing dengan educational materials tentang productivity techniques dan time management. University partnerships untuk facilitating student adoption dan feedback collection.

\textbf{Customer Success}

Onboarding optimization untuk maximizing new user retention dan feature adoption. Customer support dalam Bahasa Indonesia dengan comprehensive documentation dan tutorials. Community building initiatives untuk user engagement dan knowledge sharing.

\subsection{Saran untuk Sustainability}

\subsubsection{Technical Sustainability}

\textbf{Code Maintenance}

Regular security updates untuk dependencies dan third-party integrations. Performance monitoring implementation dengan automated alerting systems. Test coverage expansion untuk ensuring reliability during feature additions. Documentation maintenance untuk facilitating knowledge transfer dan onboarding.

\textbf{Infrastructure Optimization}

Cost optimization strategies untuk cloud services dengan efficient resource utilization. Green hosting considerations untuk environmental responsibility dan sustainability. Monitoring dan alerting systems untuk proactive issue detection. Backup dan disaster recovery planning untuk business continuity.

\subsubsection{Business Sustainability}

\textbf{Revenue Diversification}

Enterprise licensing opportunities untuk B2B market expansion. API monetization untuk enabling third-party developer ecosystem. Consulting services untuk organizations requiring custom implementations. Training programs development untuk productivity methodologies dan best practices.

\textbf{Community Building}

Open source contributions untuk fostering ecosystem growth dan developer engagement. Developer community platform untuk sharing extensions dan integrations. User feedback loops implementation untuk continuous product improvement. Partnership ecosystem development dengan complementary productivity tools.

Penelitian ini menunjukkan bahwa pengembangan sistem penjadwalan berbasis AI tidak hanya technically feasible tetapi juga memberikan significant value bagi users dalam meningkatkan produktivitas dan goal achievement. Dengan proper implementation dan continuous improvement, sistem ini memiliki potensi untuk creating sustained positive impact dalam domain personal productivity management dan dapat menjadi foundation untuk future innovations dalam AI-powered productivity applications.

% ===============================
% DAFTAR PUSTAKA
% ===============================
\newpage
\addcontentsline{toc}{chapter}{DAFTAR PUSTAKA}
\begin{center}
{\Large\bfseries DAFTAR PUSTAKA}
\end{center}

\vspace{1cm}

\begin{thebibliography}{99}

\bibitem{adams2020}
Adams, R. J., \& Thompson, K. L. (2020). Digital productivity tools and student performance: A longitudinal study. \textit{Computers \& Education}, 145, 103-114.

\bibitem{anthropic2024}
Anthropic. (2024). \textit{Claude AI API Documentation}. Retrieved from https://docs.anthropic.com/

\bibitem{brown2021}
Brown, S. M., Davis, P. R., \& Wilson, A. K. (2021). Artificial intelligence in personal time management: Current trends and future directions. \textit{International Journal of Human-Computer Studies}, 149, 102-115.

\bibitem{indonesia2023}
Badan Pusat Statistik. (2023). \textit{Statistik Telekomunikasi Indonesia 2023}. Jakarta: BPS.

\bibitem{johnson2022}
Johnson, M. R., \& Lee, S. Y. (2022). Time management strategies among university students: A cross-cultural analysis. \textit{Educational Psychology Review}, 34(2), 287-305.

\bibitem{nextjs2024}
Next.js Team. (2024). \textit{Next.js Documentation: Building Full-Stack Web Applications}. Retrieved from https://nextjs.org/docs

\bibitem{react2024}
Meta Open Source. (2024). \textit{React Documentation: The Library for Web and Native User Interfaces}. Retrieved from https://react.dev/

\bibitem{russell2020}
Russell, S., \& Norvig, P. (2020). \textit{Artificial Intelligence: A Modern Approach} (4th ed.). Boston: Pearson.

\bibitem{smith2023}
Smith, A. B., \& Garcia, C. M. (2023). Mobile application design for productivity enhancement: A user experience perspective. \textit{International Journal of Human-Computer Interaction}, 39(8), 1634-1647.

\end{thebibliography}

\end{document}